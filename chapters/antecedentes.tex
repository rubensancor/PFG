En este capitulo se hablará sobre el cliente que pide el proyecto y el estado del arte del mismo.

\section{Demandante del proyecto}
En esta sección se hablará el demandante del proyecto, en este caso es la ONG Alboan.

\subsection{Alboan}

La actividad de Alboan\cite{alboan} comenzó en 1994, asumiendo las iniciativas de voluntariado
internacional que ya estaban en marcha, pero fue en 1996 cuando se configuró
jurídicamente bajo la figura de Fundación. Es la ONG promovida por la Compañía de Jesús en la Provincia de Loyola (País Vasco y Navarra). El nombre, en euskera, Alboan, quiere reflejar el arraigo a la cultura de la tierra vasca y el espíritu de la entidad: estar al lado de las personas más excluidas, junto a organizaciones y centros educativos.\\

Su logo (Figura \ref{logo}) visibiliza  el rol de bisagra y puente que quería jugar la institución para poner en relación dos mundos que en realidad son uno solo. Su misión: ser plataforma de encuentro de personas y organizaciones de aquí y allá que quisieran comprometerse en la construcción de un mundo mejor.\\

Hoy Alboan cuenta con 7000 personas entre contratados, voluntarios y entidades que le permiten apoyar y acompañar a más de 200 proyectos llevados a cabo por 105 organizaciones
aliadas que impactan directamente en las vidas de más de 600.000 personas en todo
el mundo.

\figuraSinMarco{0.5}{imgs/alboan.jpg}{Logo de Alboan}{logo}{}

Trabaja por la construcción de una ciudadanía global que denuncie las injusticias que provocan desigualdad en el mundo, construya una cultura que promueva el bien común y transforme las estructuras generadoras de pobreza a nivel local y global. Para lograrlo, se une en red con personas y grupos de todo el mundo.\\

La colaboración de Alboan se centra en 4 temáticas principales en las que enmarca todos sus proyectos y labores de ayuda: Educación de calidad, Desarrollo económico-productivo sostenible y equitativo, Acción humanitaria en crisis recurrentes y Democracia a favor de las personas excluidas.\\

Todas las actividades que realiza la ONG incluyen 3 ejes transversales que otorgan a los proyectos el carisma que Alboan quiere dar:
\begin{itemize}
	\item La espiritualidad como dimensión en el horizonte de desarrollo humano.
	\item El reconocimiento de las desigualdades entre mujeres y hombres y el compromiso con la equidad de género.
	\item La participación ciudadana para la incidencia social y política.
\end{itemize}

\section{Estado del arte}
En este apartado se hace un análisis de las diferentes opciones que hay para cubrir las necesidades citadas anteriormente.
\subsection{Crowdfunding}
“Cooperación colectiva, llevada a cabo por personas que realizan una red para conseguir dinero u otros recursos, se suele utilizar Internet para financiar esfuerzos e iniciativas de otras personas u organizaciones”\cite{crowdw}\\

En el crowdfundig hay diferentes tipos de ayudas y de sistemas creados para financiar los proyectos que la gente sube a las plataformas: basado en donaciones (migranodearena, teaming...), basados en recompensas(Goteo, Verkami...) y para préstamos(ECrowd!) entre otros.

\subsubsection{Basado en donaciones}
El usuario realiza una donación para ayudar a un proyecto que le gusta y no recibe ningún beneficio económico en retorno (solo la satisfacción de hacer algo bueno y de apoyar un proyecto que le emociona). Son proyectos solidarios con un impacto social grande.
\begin{itemize}
	\item Migranodearena.org
	\item Teaming
\end{itemize}

\subsubsection{Basado en recompensas}
Se trata de aportar dinero a un proyecto y se recibe a cambio una recompensa. Los proyectos pueden consistir en crear y diseñar un nuevo producto o financiar un proyecto cultural (película, festival, musical). Es el modelo de crowdfunding más utilizado y conocido. En cada proyecto se puede elegir entre un rango de recompensas en función de la cantidad de dinero que se aporte.
	
\begin{itemize}
	\item Goteo
	\item Verkami
\end{itemize}
		
\subsubsection{Para prestamos}
El usuario realiza una inversión en un proyecto o en una empresa de la plataforma y recibe un retorno económico en forma de intereses. La empresa contrata un préstamo con la plataforma de crowdlending, que lo gestiona, y el usuario recupera el dinero invertido junto a los intereses a lo largo del tiempo según las condiciones pactadas en el préstamo.
		
\begin{itemize}
	\item ECrowd!
\end{itemize}
			
\subsection{Micro donaciones}
Las micro donaciones, como su nombre indica, son donaciones muy pequeñas realizadas con el fin de aportar una pequeña cantidad de dinero al proyecto que nos interesa. En este ámbito nos encontramos con Worldcoo\cite{worldcoo} una herramienta online con bastante recorrido y reconocimiento.
			
\begin{itemize}
	\item \textbf{Worldcoo:} \smallbreak
	Worldcoo es una empresa nacida en 2012 que pretende crear un nuevo canal de financiación mediante un widget incrustado en tiendas online. Esto permite que las miles de personas que acceden y compran en las tiendas online puedan donar una pequeña cantidad al proyecto que esa página tiene apadrinado.\\
	
	Cuentan con un extenso equipo de personas y de embajadores en varios paises del mundo, lo que les da acceso a muchos comercios online. Las empresas contactan con Worldcoo y acuerdan entre los dos la instalación del widget en su tienda online. Este widget esta relacionado con uno de los proyectos de las ONGs a las que Worldcoo ayuda.
\end{itemize}
				
Esto es exactamente lo que Alboan quiere, pero tiene algunas carencias. La empresa Worldcoo se queda un 8\% de las donaciones realizadas a cada proyecto por lo que esto empieza a disgustar a Alboan, esta empresa tampoco permite ver las personas que están donando a los proyectos por lo que no permite mantener el contacto con esas personas para hacerles conscientes de que su ayuda tiene un resultado. \\

Después de hacer un análisis de todas las opciones y ver los resultados Alboan decide crear su propia plataforma de micro donaciones mediante widgets en tiendas online.