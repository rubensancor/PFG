\smallbreak %borrar
En este capítulo hablaremos sobre el beneficiario del proyecto, en este caso es la ONG ALBOAN.

\section{Alboan}

La actividad de Alboan comenzó en 1994, asumiendo las iniciativas de voluntariado
internacional que ya estaban en marcha, pero fue en 1996 cuando se configuró
jurídicamente bajo la figura de Fundación. Es la ONG promovida por la Compañía de Jesús en la Provincia de Loyola (País Vasco y Navarra). El nombre, en euskera, Alboan, quiere reflejar el arraigo a la cultura de la tierra vasca y el espíritu de la entidad: estar al lado de las personas más excluidas, junto a organizaciones y centros educativos.\\

Su logo(Figura \ref{logo}) visibiliza  el rol de bisagra y puente que quería jugar la institución para poner en relación dos mundos que en realidad son uno solo. Su misión: ser plataforma de encuentro de personas y organizaciones de aquí y allá que quisieran comprometerse en la construcción de un mundo mejor.\\

Hoy Alboan cuenta con 7000 personas entre contratados, voluntarios y entidades que le permiten apoyar y acompañar a más de 200 proyectos llevados a cabo por 105 organizaciones
aliadas que impactan directamente en las vidas de más de 600.000 personas en todo
el mundo.

\figura{0.5}{imgs/alboan.jpg}{Logo de Alboan}{logo}{}

\subsection{Misión y Visión}

\subsubsection{Misión}

Trabaja por la construcción de una ciudadanía global que denuncie las injusticias que provocan desigualdad en el mundo, construya una cultura que promueva el bien común y transforme las estructuras generadoras de pobreza a nivel local y global. Para lograrlo, se une en red con personas y grupos de todo el mundo.\\

La colaboración de Alboan se centra en 4 temáticas principales en las que enmarca todos sus proyectos y labores de ayuda: Educación de calidad, Desarrollo económico-productivo sostenible y equitativo, Acción humanitaria en crisis recurrentes y Democracia a favor de las personas excluidas.\\

Todas las actividades que realiza la ONG incluyen 3 ejes transversales que otorgan a los proyectos el carisma que Alboan quiere dar:
\begin{itemize}
	\item La espiritualidad como dimensión en el horizonte de desarrollo humano.
	\item El reconocimiento de las desigualdades entre mujeres y hombres y el compromiso con la equidad de género.
	\item La participación ciudadana para la incidencia social y política.
\end{itemize}

\subsubsection{Visión}

Lograr una Alboan que contenga las siguientes características:
\begin{itemize}
	\item \textbf{Enraizada} en el nuevo proyecto unificado de la Compañía de Jesús a través de sus plataformas locales y territorial.
	\item \textbf{Querida} por las organizaciones y la base social con las que se alía.
	\item \textbf{Reconocida} por su valor añadido en el acompañamiento a entidades, la
	formación y la construcción de ciudadanía global.
	\item \textbf{Sostenible} gracias a un equipo comprometido y una financiación estable y
	diversificada.
	\item \textbf{Ilusionante}, por sus propuestas y su comunicación esperanzadora.
	\item \textbf{Puente} entre nuestro estilo de vida y las situaciones de frontera de
	deshumanización
\end{itemize}


\subsection{Dedicación}

Alboan tiene unos ámbitos de dedicación marcados en los que

\begin{itemize}
	\item La \textbf{Cooperación al Desarrollo} en África, Asia, Centroamérica y Sudamérica, principalmente con proyectos educativos, de promoción económica y de formación de grupos excluidos para la defensa de sus derechos.

	\item La \textbf{Educación para la solidaridad} en Euskadi y en Navarra, mediante la participación en campañas, la elaboración de materiales educativos, la promoción del comercio justo, impartiendo formaciones y talleres y asesorando a grupos y centros educativos.

	\item La \textbf{Acción Política} participando en redes y elaborando estudios e investigaciones para incidir y mejorar las políticas que afectan al desarrollo tanto locales como internacionales.


\end{itemize}
