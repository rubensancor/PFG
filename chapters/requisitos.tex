\section{Especificación de requisitos del servidor Node.js}
Los requisitos funcionales del servidor Node.js son:

\begin{itemize}
	\item \textbf{RF.0.1} El servidor debe ser capaz de albergar la página web.
	\item \textbf{RF.0.2} El servidor debe ser capaz de albergar las rutas de la página web y ofrecer el enrutamiento a cada una de estas.
	\item \textbf{RF.0.3} El servidor debe ser capaz de conectarse con la base de datos.
	\item \textbf{RF.0.4} El servidor debe ser capaz de enviar mails.
	\item \textbf{RF.0.5} El servidor debe ser capaz de alterar un PDF.
	\item \textbf{RF.0.6} El servidor debe ser capaz de crear un script y almacenarlo.
	\item \textbf{RF.0.7} El servidor debe ser capaz de ofrecer un script a páginas de terceros.
	\item \textbf{RF.0.8} El servidor debe ser capaz de ofrecer datos a una tercera aplicación.
\end{itemize}

Los requisitos no funcionales son:

\begin{itemize}
	\item \textbf{RNF0.1} Sostenibilidad: el sistema tiene que tener un mantenimiento sencillo ya que tiene conexiones con paginas de terceros lo que obliga a que el mantenimiento sea sencillo y rápido.
	\item \textbf{RNF0.2} Escalabilidad: el servidor tiene que ser escalable ya que la creación de nuevos widgets o la oferta de datos a terceras aplicaciones puede ser grande.
\end{itemize}



\section{Especificación de requisitos de la página web}

Los requisitos funcionales de la página web son:

\begin{itemize}
	\item \textbf{RF.0.1} La página web debe ser 
\end{itemize}



\section{Especificación de requisitos de la base de datos}

\section{Especificación de requisitos del sistema de visualización}