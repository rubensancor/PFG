\section{Visión general}

En este capitulo se especifican los requisitos que el proyecto debe satisfacer y que definen el funcionamiento de todo el software que compone este proyecto. Para una mejor compresión de los mismos, se dividen en los siguientes bloques:

\begin{itemize}
	\item \textit{Especificación de requisitos del servidor Node.js:} en esta sección se recogen los requisitos que debe de satisfacer el servidor, este es el encargado de soportar todo el sistema.
	\item \textit{Especificación de requisitos de la página web:} en esta sección se recogen los requisitos que debe satisfacer la página web del proyecto.
	\item \textit{Especificación de requisitos del widget:} en esta sección se recogen los requisitos que debe satisfacer el widget del proyecto, el elemento que estará incrustado en las tiendas online.
	\item \textit{Especificación de requisitos de la base de datos:} en esta sección se recogen los requisitos que debe satisfacer la base de datos, esta albergará los datos de las donaciones y donantes.
	\item \textit{Especificación de requisitos del sistema de visualización:} en esta sección se recogen los requisitos que debe satisfacer el sistema de visualización, el encargado de crear gráficas interactivas para la visualización y estudio de los datos.
\end{itemize}

\section{Especificación de requisitos del servidor Node.js}
Los requisitos funcionales del servidor Node.js son:

\begin{itemize}
	\item \textbf{RF.0.1} El servidor debe ser capaz de albergar la página web.
	\item \textbf{RF.0.2} El servidor debe ser capaz de albergar las rutas de la página web y ofrecer el enrutamiento a cada una de estas.
	\item \textbf{RF.0.3} El servidor debe ser capaz de conectarse con la base de datos.
	\item \textbf{RF.0.4} El servidor debe ser capaz de enviar mails.
	\item \textbf{RF.0.5} El servidor debe ser capaz de alterar un PDF.
	\item \textbf{RF.0.6} El servidor debe ser capaz de crear un script y almacenarlo.
	\item \textbf{RF.0.7} El servidor debe ser capaz de ofrecer un script a páginas de terceros.
	\item \textbf{RF.0.8} El servidor debe ser capaz de ofrecer datos a una tercera aplicación.
\end{itemize}

Los requisitos no funcionales son:

\begin{itemize}
	\item \textbf{RNF.0.1} Mantenibilidad: el sistema tiene que tener un mantenimiento sencillo ya que tiene conexiones con paginas de terceros lo que obliga a que el mantenimiento sea sencillo y rápido.
	\item \textbf{RNF.0.2} Escalabilidad: el servidor tiene que ser escalable ya que la creación de nuevos widgets o la oferta de datos a terceras aplicaciones puede ser grande.
\end{itemize}



\section{Especificación de requisitos de la página web}

Los requisitos funcionales de la página web son:

\begin{itemize}
	\item \textbf{RF.1.1} La página web debe ofrecer un wizard para la creación de nuevos. widgets
	\item \textbf{RF.1.2} La página web debe ofrecer un formulario para la obtención de un certificado de donación.
	\item \textbf{RF.1.3} La página web debe ofrecer un sistema para comunicarse con el soporte del proyecto.
\end{itemize}

Los requisitos no funcionales de la página web son:

\begin{itemize}
	\item \textbf{RNF.1.1} La página web debe informar sobre el proyecto.
	\item \textbf{RNF.1.2} La página web debe informar sobre los proyectos de la ONG.
	\item \textbf{RNF.1.3} Accesibilidad: La página web tiene que ser responsiva para ser adaptable en diferentes dispositivos.
	\item \textbf{RNF.1.4} Interfaz: la página web debe tener un diseño simple para facilitar la navegación.
	\item \textbf{RNF.1.5} Escalabilidad: la página tiene que ser escalable ya que se tienen que poder añadir nuevos proyectos a ella.
\end{itemize}

\section{Especificación de requisitos del widget}

Los requisitos funcionales del widget son:

\begin{itemize}
	\item \textbf{RF.2.1} El widget debe permitir la donación fija o variable de una cantidad de dinero.
	\item \textbf{RF.2.2} El widget debe ser de fácil implantación por parte de la tienda online.
\end{itemize}

Los requisitos no funcionales del widget son:

\begin{itemize}
	\item \textbf{RNF.2.1} El widget debe informar sobre el proyecto al que va destinado.
	\item \textbf{RNF.2.2} Accesibilidad: el widget debe ser responsivo para que pueda añadirse en cualquier tienda.
	\item \textbf{RNF.2.3} Interfaz: el widget debe ser 100\% personalizable.
	\item \textbf{RNF.2.4} Disponibilidad: el widget debe estar disponible en todo momento para su uso por parte de los comercios online.
\end{itemize}

\section{Especificación de requisitos de la base de datos}

Los requisitos funcionales de la base de datos son:

\begin{itemize}
	\item \textbf{RF.3.1} La base de datos debe almacenar los datos de las donaciones.
	\item \textbf{RF.3.2} La base de datos debe almacenar los dados de los donantes.
	\item \textbf{RF.3.3} La base de datos debe proporcionar los datos que se le pidan.
\end{itemize}

Los requisitos no funcionales de la base de datos son:

\begin{itemize}
	\item \textbf{RNF.3.1} Rendimiento: la base de datos debe almacenar y proporcionar los datos en un tiempo razonable.
	\item \textbf{RNF.3.2} Seguridad: la base de datos debe ser segura para no poner en peligro los datos de los donantes.
	\item \textbf{RNF.3.3} Disponibilidad: La base de datos debe estar disponible para almacenar las donaciones e información de donantes.
\end{itemize}

\section{Especificación de requisitos del sistema de visualización}

Los requisitos funcionales del sistema de visualización son:

\begin{itemize}
	\item \textbf{RF.4.1} El sistema de visualización debe ofrecer gráficos interactivos.	
	\item \textbf{RF.4.2} El sistema de visualización debe conectarse con el servidor para adquirir los datos.
\end{itemize}

Los requisitos no funcionales del sistema de visualización son:

\begin{itemize}
		\item \textbf{RNF.4.1} Escalabilidad: el sistema de visualización de datos debe ser escalable ya que los datos pueden crecer y la manera de mostrarlos cambiar.
		\item \textbf{RNF.4.2} Interfaz: el sistema de visualización de datos debe tener una interfaz intuitiva que permita una buena navegación por los gráficos.
\end{itemize}