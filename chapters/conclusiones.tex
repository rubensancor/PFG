Este capitulo contiene las conclusiones que se han obtenido después de haber desarrollado el proyecto y las posibles acciones que se podrían tomar en el futuro para mejorarlo o integrarlo con el sistema que existe en Alboan.

\section{Conclusiones}
Al comienzo del proyecto este se presentaba sin una definición clara ya que Alboan sabia que quería un nuevo sistema para las micro donaciones y que quería que estuviese implementado con un widget que soportase las micro donaciones. Junto a ellos se consiguió definir que la solución incluiría una página web que publicitase el proyecto. Finalmente y cuando el proyecto tuvo mas forma se decidió añadirle el sistema de visualización de datos para crear los gráficos y estadísticas que Alboan posteriormente podría utilizar. Es por esto que el proyecto ha tenido unos requisitos un poco cambiantes y se le ha permitido al cliente cambiar las historias de cliente.\\

El desarrollo ha sido fluido y el proyecto se ha ido completando con normalidad. En varias ocasiones ha habido dudas sobre como desarrollar algunas de las funcionalidades pero con ayuda del director de proyecto, que me puso en contacto con personas mas experimentadas en el tema, pude resolverlas sin ningún problema.\\

Gracias al proyecto he tenido la oportunidad de aprender a gestionar proyectos por lo que me he interesado por las diferentes metodologías que existen. En cuanto a la decision de las tecnologias he decido interesarme mas por las nuevas metodologías ágiles que me han permitido aprender nuevos modelos de gestion de proyectos. Esta metodologia, scrum, ya las habia aplicado en una asignatura de clase, pero en un entorno muy controlado, al llevarla a la practica me he dado cuenta que la teoria no se aplica del todo en la practica.\\

Por otra parte el proyecto me ha exigido formación en muchas tecnologias que no habia utilizado anteriormente. Gracias a este he tenido que que investigar en las tecnologias web actuales y me ha exigido formarme tanto en Node.js como en los paquetes que ofrece mediante \textit{npm}. Por otra parte, de las tecnologías básicas del desarrollo web también he tenido que formarme ya que al intentar desarrollar en HTML5 o utilizar un precompilador para las hojas de estilo he adquirido mucha formación en estos temas.\\

En conclusión, el proyecto ha sido una oportunidad para prepararme para lo que puede ser mi futuro laboral, tanto en la gestión de proyectos como en el desarrollo de los diferentes elementos del sistema completo. Al tener que interpretar varios roles en el desarrollo del proyecto he visto cuales son las exigencias y alcance de cada uno, lo que me ha permitido discernir sobre varias opciones de mi futuro laboral.

\section{Lineas futuras}
El avance mas importante que se podría hacer de aquí en adelante seria conseguir integrar el proyecto con la infraestructura de Alboan. Alboan tiene subcontratada a una empresa que le lleva toda la infraestructura tecnológica y la comunicación con ellos no ha sido buena ya que no estaban seguros de si integrar el proyecto por miedo a que Alboan no lo aceptase del todo. \\

A parte de esta mejora principal se podrán implementar nuevas lineas que permitirán que el proyecto sea de mas calidad:

\begin{itemize}
	\item Implementar i18n en la página web para que sea mas accesible mediante \textit{i18n-express}
	\item Implementar un sistema de comentarios que los demás usuarios puedan ver para recoger las opiniones.
\end{itemize}