\section{Visión general}
Este capítulo contiene los aspectos más significativos sobre la implementación del sistema y sus funcionalidades.El contenido central de este capitulo se centra en explicar que pasos se han dado a la hora de desarrollar el sistema y como se ha organizado este, simplificando la labor de comprensión del sistema para terceros. Por último, al principio del capitulo se explicarán las herramientas que se han utilizado a la hora de desarrollar el sistema. 
\section{Entorno de desarrollo}
Las herramientas que se han utilizado para facilitar el desarrollo y la arquitectura del proyecto han sido las siguientes.

\subsection{Atom}
Atom se define a si mismo como ``el editor de textos hackable del siglo XXI''. Dentro de este editor de texto o entorno de desarrollo integrado (IDE), ya que cumple todos los requisitos de uno de estos, se encuentran una serie de funcionalidades y características que lo hacen una de las herramientas mas potentes del mercado. Dentro de las funcionalidades que ofrece esta herramienta se encuentran los paquetes que la comunidad crea y desarrolla, gracias a estos paquetes el programa a conseguido implementar la corrección y sintaxis de muchos lenguajes. Por otra parte, este editor de textos es totalmente personalizable ya que esta desarrollado con tecnologias web y ofrece una IU totalmente modificable mediante CSS/Less. Por ultimo, y no menos importante, Atom esta desarrollado por GitHub, por lo que es open-source y ofrece la posibilidad de ayudar en el desarrollo del código de la aplicación.

\subsection{Brackets}
Brackets es un editor de texto moderno enfocado en el diseño de la parte visual de las aplicaciones web. La funcionalidad mas destacada de este software es la vista previa que permite al desarrollador ver en directo como quedarán los cambios que realice en el código, directamente en la página web, esto ahorra tiempo al no tener que recargar la web constantemente.Brackets también ofrece un editor interno que permite navegar por todos los archivos CSS que una etiqueta implemente, lo que aligera la carga de estar cambiando entre las diferentes hojas de estilo. Por ultimo, entre sus funcionalidades tambien destaca el soporte a los preprocesadores, brackets permite hacer cambios directamente en los archivos de las hojas de estilo de los preprocesadores y ver esos cambios directamente, sin tener que compilarlos.

\subsection{Git}
Git es un sistema de control de versiones distribuido gratuito y de código abierto que está diseñado para manejar desde pequeños proyectos particulares a proyectos de grandes organizaciones con una gran velocidad y eficiencia. Además es muy fácil de aprender, tiene una excelente documentación y al ser usado por muchísimos desarrolladores tiene una gran comunidad de usuarios dispuestos a resolver cualquier problema.

\section{Implementación del servidor}
El servidor esta implementado en Node.js con la ayuda de varios paquetes creados por la comunidad. El servidor esta diferenciado en dos archivos diferentes. El archivo \textit{server.js} contiene la configuracion de los paquetes que va a utilizar la aplicacion y el puerto en el que se va a desplegar. El archivo que contiene toda la funcionalidad del sistema es \textit{api.js}, en este archivo estan definidas las rutas y las funciones del proyecto.\\

Lo primero en lo que fijarse al entrar en el archivo \textit{server.js} es en los paquetes que tiene como requerimientos y los que utiliza. Entre los paquetes y usos mas destacables esta el requerimiento del otro archivo mas importante del proyecto, \textit{api.js}, el uso del framework express y la carga de EJS como el motor visual del proyecto. Por otra parte, cabe destacar el puntero (Listado ) que se crea para indicar al sistema donde van a estar alojados todos los recursos públicos del sistema.\\

\codigofuente{Java}{Puntero a la carpeta public}{src/puntero.js}

En el archivo \textit{api.js} se encuentra toda la funcionalidad del servidor separada por las rutas que este ofrece a la página web. En esta seccion se hablará de como se han implementado las diferentes funcionalidades.

\subsection{Como añadir una nueva ruta}
Para crear una nueva ruta dentro del servidor solo habria que añadir una nueva respuesta a una solicitud de la parte del cliente (Listado ). En el direccionamiento existen 4 tipos de solicitudes desde la parte servidora, estas serian: GET, POST, PUT y DELETE. Posteriormente habria que indicar a que ruta harian esta solicitud, en el codigo marcado con /RUTA. Finalmente se entraria en la funcion en la que se desarrollaria todo el codigo que el programador haya especificado.\\ 

\codigofuente{Java}{Ejemplo de ruta con Express}{src/ruta.js}

\subsection{Conexión con la base de datos}

Lo primero a destacar es la conexión con la base de datos, esta esta implementada con mediante el paquete mongojs. La conexión es muy sencilla (Listado) y al solo tener una colección nos conectamos directamente a ella. La  base de datos al ser MongoDB se despliega en el puerto 27017, en caso de estar desplegada solo habria que cambiar ese dato. El nombre de la base de datos es colmena y la coleccion donaciones.\\

\codigofuente{Java}{Conexión a la base de datos mediante mongojs}{src/mongojs.js}

\subsection{Configuracion de Nodemailer}
Nodemailer\cite{keylist}, es el paquete que nos permite enviar los mails a las personas que donan. La configuración de este paquete es muy sencilla y nos permite elegir servicio con el que vamos a enviar los mails, en este caso seria Gmail y solo habria que añadir el mail y la contraseña de la cuenta que queramos utilizar.

\codigofuente{Java}{Configuracion de Nodemailer}{src/nodemailer.js}

Las funcionalidades que ofrece el servidor son REST, es decir, esperan a que una peticion entrante para reaccionar y cuando acaban cortan la comunicacion. 

\subsection{Codificacion de la función de recepción de donaciones}
El sistema sistema de recepcion de donaciones esta codificado de manera que al recibir una solicitud de donacion desde el widget, la funcion, mediante un algoritmo, calcula un id de donacion unico para cada usuario. Posteriormente empaqueta este numero y lo envia a la base de datos junto con los datos de la donación. Por ultimo genera un email con el id de donacion y se lo envia a la persona que ha realizado la compra.\\

El algoritmo que genera los numeros de donación se ha creado utilizando un metodo en el que es practicamente imposible generar un numero igual. El algortimo comienza capturando el tiempo en el que llega la peticion, por lo que este numero solo puede ser igual si la peticion llega en el mismo momento exacto. El numero que se genera tiene 13 cifras. Posteriormente se genera un numero aleatorio entre 0 y 100 y se suma al número obtenido anteriormente. Las posibilidades de generar un número igual solo serian significativas, un 1\%, si la peticion llegase exactamente en el mismo momento.

\codigofuente{Java}{Algoritmo de generacion de numeros de donacion}{src/donacion.js}

\subsection{Codificacion de la creacion de certificados}
El sistema de creacion de certificados esta apoyado en el paquete pdffiller que permite al sistema completar el formulario de un pdf modelo de un certificado de donacion oficial de la ONG Alboan. \\

En primer lugar el sistema comprueba que la donación no este utiliza consultando en la base de datos por el atributo \textit{usada} del documento. Una vez comprobada que esto no es asi redirecciona al usuario al formulario en el que debe añadir los datos con los que quiere que se genere el formulario.\\

Una vez enviado el formulario el servidor se encarga de rellenar el PDF, enviar el mail con este al usuario y de actualizar la base de datos para que la donacion se marque como usada y se añadan los datos del donante. Todas estas acciones se realizan en conjunto y en caso de haber algun error se paralizará la funcion.

\subsection{Codificacion del sistema de guardado de widgets}
El sistema de guardado de widgets es muy sencillo ya que desde la página web se generar con un formato fijo. Esta funcion recibe desde la página web el script con los datos que el usuario a establecido y le añade unas funciones fijas que comparten todos los widgets. Una vez creado el widget minimizado, es decir, en una sola linea para ahorrar espacio, se intenta guardar en la carpeta donde estarán alojados todos los widgets. En caso de que el nombre del widgets ya exista, se comunicará al usuario. \\

El nombre que se le aplicará al widget seguirá el siguiente formato: Nombre de la empresa + \_colmena.js. \\


\codigofuente{Java}{Codigo del sistema de guardado de widgets}{src/autosave.js}

\subsection{Como utilizar la API para consultar datos}
El servidor tambien ofrece una API para consultar diferentes datos y hacer nuestras propias estadisticas sin tener que depender de la aplicacion del proyecto. Esta API ofrece los datos en JSON sin formatear y directos de la base de datos. La API estará restringida para los datos sensibles.\\

En el siguiente ejemplo se nos ofrece la posibilidad de entrar en la siguiente ruta: IPSERVIDOR/date/:year/:month. Cambiando \textit{year} y \textit{mont} por el año y el mes que queramos respectivamente, se nos ofrecera un json con todas las donaciones que ha habido ese mes.

\codigofuente{Java}{Ejemplo de ruta de la API}{src/apidatos.js}

\section{Implementación de la página web}
La página web esta implementada sobre una plantilla de Bootstrap lo que ha permitido que el desarrollo se haga mas agil. En cambio, la plantilla ha sido alterada para que el diseño encaje con el de la ONG, por lo que de la plantilla original solo queda la estructura. Dentro del código de la pagina web tambien se han incluido varias librerias que permiten una experiencia mas inmersiva y con mas funcionalidades. \\

La página web esta dividida en plantillas \textit{ejs} que permiten modularizar la estructura de la página y hacer la inclusión de nuevos modulos o pantallas mas sencilla. La plantilla principal de la página sera la que incluya todas las demas y las vaya alterando según donde se encuentre el usuario. En la plantilla principal o \textit{layout} se añadiran los diferentes paquetes o librerias que todas las plantillas deban implementar. Es por esto que la primera vez que se entre a la página web se cargaran todos los paquetes y posteriormente la navegación sera mas fluida.\\

La página web cuenta en su parte superior con un \textit{scrollspy}, es decir, la barra superior es un acceso directo a las diferentes secciones que tiene la página web que va siguiendo al usuario a medida que se va moviendo por la página web, de manera que siempre esta disponible. Esta barra esta diseñada de manera que sea responsiva, cuando el dispositivo desde el que se navegue sea demasiado pequeño, esta se unificará y se convertira en un menu desplegable.\\

Las diferentes secciones de la página web se han desarrollado mediante la etiqueta \textit{section} (Listado 6.1). Esta etiqueta fue implementada en HTML5, la última version de esta tecnologia web. Estas secciones estan referenciadas desde la barra superior que hemos citado anteriormente, por lo que los usuarios podran moverse mas rapidamente por la página web.\\


\codigofuente{HTML}{Uso de section en la página web}{src/section.html}

Dentro de la página web hay varios botones que al pulsarlos despliegan diferentes ventanas las que hay desde información sobre los proyectos a los que se apoya con el proyecto hasta un asistente para crear tu propio widget. Estas ventanas, denominadas modales, estan ocultas en todo momento hasta que se les llama desde con algun hipervinculo en un boton. Estas ventanas (Listado 6.2) implementan varias clases del framework Bootstrap, necesarios para conseguir que no esten visibles y hagan un efecto de entrada. El atributo \textit{tabindex} designa cual sera el orden del foco a la hora de pulsar el boton tabulacion. El atributo \textit{role=dialog} se utiliza para marcar que el elemento esta separado de la pagina principal y es un dialogo o ventana. Por ultimo, el atributo \textit{aria-hidden} indica que el elemento no sera visible por el usuario hasta que se llame.\\

\codigofuente{HTML}{Código de una ventana Modal}{src/modal.html}

\subsection{Codificación del sistema de expedición de certificados de donación}
El sistema de expedición de certificados de donación se basa en un campo en el que introducir la donación. Una vez introducida el numero de donación, unico para cada una y creado mediante un algoritmo explicado en la seccion anterior. El número de donación se consulta en la base de datos mediante un proceso explicado en la seccion anterior.\\

Una vez pasado la introduccion del codigo de donacion nos aparece un formulario que ha sido implementado mediante clases del framework Bootstrap. Lo mas destacable se encuentra dentro del script \textit{form.js} en el que se implementan un par de funciones que le otorgan logica al formulario.\\

Dentro de las dos funciones que alberga este script se encuentra la que valida el formulario, comprobando que todos los campos esten completos y sobre todo que el mail este correcto, ya que si el mail esta mal el certificado no llegará al usuario. Para esta comprobacion se ha consultado en internet una expresion regular y se ha comprobado su eficacia en un testeador de expersiones regulares para emails online. \\

\codigofuente{HTML}{Expresión regular que comprueba los emails}{src/expr.html}

Una vez se valida todo el formulario, se empaquetan todos los datos en JSON y se envian al servidor para que este los guarde en la base de datos. En esta transaccion se utiliza la tecnologia AJAX\cite{wsdl14} que permite la comunicacion entre el servidor y la pagina web.\\

\codigofuente{Java}{Código del envio de información del certificado al servidor}{src/certificado.js}

\subsection{Codificación del asistente de creación de widgets}
El sistema de creacion de widgets es similar al de expedición de certificados en la parte visual. Al clicar en un boton se desplegará una ventana modal y hay estará el asistente para la creacion de widgets. En este asistente no se creará el widget real, es una demo, para acceder al asistente real habrá que contactar con el soporte del proyecto.\\

El asistente consta de un formulario y un widget interactivo que va cambiando dependiendo de las modificaciones que se realicen en el formulario. Esto es posible gracias al script \textit{widget.js} en el que esta desarrollada toda la logica de este asistente. Para que este asistente funcione tambien se han incluido otras librerias como la de seleccion de color\cite{wsdl15} ofrecida por un desarrollador de la comunidad.\\

Dentro del script nombrado en el anterior parrafo se encuentran una serie de funciones que alteran esteticamente el widget, en este ejemplo (Listado 6.5) se muestra la funcion para cambiar el color del widget. En este caso se utiliza tambien la libreria nombrada anteriormente que ofrece una paleta de colores al usuario.  En esta función podemos comprobar como al ir cambiando el color en el selector de color el atributo CSS de color de fondo del widget irá cambiando.\\

\codigofuente{Java}{Funcion para cambiar el color del widget}{src/cambiarcolor.js}

Una vez diseñado el widget a nuestra manera otras 2 funciones seran las encargadas de recoger toda la informacion del diseño del widget y de enviarselo al servidor para que este lo guarde y este disponible automaticamente. La informacion del widget se envia en JSON y se envía al servidor de la misma manera que se enviaba la informacion del certificado, mediante la tecnologia AJAX (Listado 6.6).\\

\codigofuente{Java}{Un fragmento de la función encargada de enviar el widget al servidor}{src/enviowidget.js}

\section{Implementación del widget}
El widget esta implementado de la manera mas sencilla y con las tecnologias menos pesadas posibles para no suponer una carga para la tienda que lo implemente. El widget estará alojado en el servidor y la tienda implementará una llamada a este para que este se integre en la tienda cada vez que un comprador entra a ella.\\

El widget consta de dos funciones que permiten que el importe de donacion varie segun las opciones que elija el usuario y otra que es la encargada de enviar los datos al servidor para que este recoja las donaciones. Finalmente el widget tiene el diseño creado por la empresa.\\

Como implementacion a destacar se encuentra la manera de codificar el widget. Se ha desarrollado de manera que no necesite una hoja de estilos anexa para funcionar, ya que él mismo implementa el estilo dentro del HTML. Todo el widget esta escrito en un String para que este luego sea inyectado en el comercio online. En la muestra de codigo que se puede ver a continuacion (Listado 6.14) no se ha añadido todo el codigo ya que la extension y el formato de este, al estar escritos sobre un String, hacia muy dificil su lectura.

\codigofuente{HTML}{Codigo de un widget}{src/widget.html}

\section{Implementación del sistema de visualización}
El sistema de visualizacion de datos esta implementado con la libreria D3.js