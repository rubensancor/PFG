\section{Visión general}
Este capítulo contiene los aspectos más significativos sobre la implementación del sistema y sus funcionalidades.El contenido central de este capitulo se centra en explicar que pasos se han dado a la hora de desarrollar el sistema y como se ha organizado este, simplificando la labor de comprensión del sistema para terceros. Por último, al principio del capitulo se explicarán las herramientas que se han utilizado a la hora de desarrollar el sistema. 
\section{Entorno de desarrollo}
Las herramientas que se han utilizado para facilitar el desarrollo y la arquitectura del proyecto han sido las siguientes.

\subsection{Atom}
Atom se define a si mismo como ``el editor de textos hackable del siglo XXI''. Dentro de este editor de texto o entorno de desarrollo integrado (IDE), ya que cumple todos los requisitos de uno de estos, se encuentran una serie de funcionalidades y características que lo hacen una de las herramientas mas potentes del mercado. Dentro de las funcionalidades que ofrece esta herramienta se encuentran los paquetes que la comunidad crea y desarrolla, gracias a estos paquetes el programa a conseguido implementar la corrección y sintaxis de muchos lenguajes. Por otra parte, este editor de textos es totalmente personalizable ya que esta desarrollado con tecnologias web y ofrece una IU totalmente modificable mediante CSS/Less. Por ultimo, y no menos importante, Atom esta desarrollado por GitHub, por lo que es open-source y ofrece la posibilidad de ayudar en el desarrollo del código de la aplicación.

\subsection{Brackets}
Brackets es un editor de texto moderno enfocado en el diseño de la parte visual de las aplicaciones web. La funcionalidad mas destacada de este software es la vista previa que permite al desarrollador ver en directo como quedarán los cambios que realice en el código, directamente en la página web, esto ahorra tiempo al no tener que recargar la web constantemente.Brackets también ofrece un editor interno que permite navegar por todos los archivos CSS que una etiqueta implemente, lo que aligera la carga de estar cambiando entre las diferentes hojas de estilo. Por ultimo, entre sus funcionalidades tambien destaca el soporte a los preprocesadores, brackets permite hacer cambios directamente en los archivos de las hojas de estilo de los preprocesadores y ver esos cambios directamente, sin tener que compilarlos.

\subsection{Git}
Git es un sistema de control de versiones distribuido gratuito y de código abierto que está diseñado para manejar desde pequeños proyectos particulares a proyectos de grandes organizaciones con una gran velocidad y eficiencia. Además es muy fácil de aprender, tiene una excelente documentación y al ser usado por muchísimos desarrolladores tiene una gran comunidad de usuarios dispuestos a resolver cualquier problema.

\section{Implementación del servidor}

\section{Implementación de la página web}
La página web esta implementada sobre una plantilla de Bootstrap lo que ha permitido que el desarrollo se haga mas agil. En cambio, la plantilla ha sido alterada para que el diseño encaje con el de la ONG, por lo que de la plantilla original solo queda la estructura. \\

La página web esta dividida por plantillas que permiten 
\section{Implementación del widget}

\section{Implementación de la base de datos}