\section{Presentación del documento}

El presente documento recoge la definición de objetivos del proyecto que realizará el alumno Rubén Sánchez para la ONG ALBOAN. En el proyecto consta de todo el sistema necesario para crear un sistema de micro donaciones mediante un widget 5 en cualquier tienda online. El sistema consta de: una página web, una base de datos, el widget y un servidor que da soporte a todo. Además, el sistema cuenta con una funcionalidad añadida por la que se puede crear gráficas de visualización.\\

El proyecto se desarrolla para la ONG Alboan la cual al ver el cambio que se da en la sociedad en materia de solidaridad y que las grandes donaciones han sufrido una gran caída decide que la manera de recoger las donaciones tiene que cambiar. Alboan, al explorar las oportunidades y ver que la única opción disponible cobra a las ONGs un porcentaje de las donaciones decide crear un sistema de micro donaciones gratuito que cualquier empresa pueda añadir a su tienda online y así fomentar las donaciones a pequeña escala, consiguiendo así que el 100\% del dinero vaya al destino final.\\

Los principales capítulos del documento son los siguientes:

\begin{itemize}
	\item \textbf{Introducción} \smallbreak
	 En este apartado se incluye una visión global del documento y del proyecto que se va a llevar a cabo.
	\item \textbf{Demandante del proyecto} \smallbreak
	 En este apartado se hará una descripción sobre la empresa demandante.
	 \item \textbf{Situación de partida} \smallbreak
	 En este apartado se trata la motivación por la que se va a desarrollar el proyecto, con esto también se analizan las posibles soluciones dadas para el problema y por últimos los objetivos que se quieren conseguir.
	\item \textbf{Definición del proyecto}\smallbreak
	Este es el apartado más importante del documento, en él se describe el producto final que se va a generar y la metodología utilizada para desarrollarlo.
	\item \textbf{Condiciones de ejecución}\smallbreak
	En este apartado se describen los materiales utilizados para desarrollar el proyecto, el hardware necesario para su puesta en marcha y el equipo que lo desarrollará.
	\item \textbf{Presupuestos y tiempos} \smallbreak
	En este apartado se describen los gastos que puede tener el proyecto en materia de tareas y equipo utilizado para desarrollarlo.
\end{itemize}
