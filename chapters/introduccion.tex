\section{Presentación del documento}

El presente documento recoge proyecto desarrollado por el alumno Rubén Sánchez para la ONG Alboan. En el proyecto consta de todo el sistema necesario para crear un sistema de micro donaciones mediante un widget en cualquier tienda online. El sistema consta de: una página web, una base de datos, el widget y un servidor que da soporte a todo. Además, el sistema cuenta con una funcionalidad añadida por la que se puede crear gráficas de visualización.\\

El proyecto se desarrolla para la ONG Alboan la cual al ver el cambio que se da en la sociedad en materia de solidaridad y que las grandes donaciones han sufrido una gran caída decide que la manera de recoger las donaciones tiene que cambiar. Alboan, al explorar las oportunidades y ver que la única opción disponible cobra a las ONGs un porcentaje de las donaciones decide crear un sistema de micro donaciones gratuito que cualquier empresa pueda añadir a su tienda online y así fomentar las donaciones a pequeña escala, consiguiendo así que el 100\% del dinero vaya al destino final.\\

Los principales capítulos del documento son los siguientes:

\begin{itemize}
	\item \textbf{Antecedentes y justificación}\smallbreak
	En este capitulo se especifica el cliente del proyecto junto con el estado del arte sobre las tecnologias creadas en este ambito.
	\item \textbf{Objetivos del proyecto}\smallbreak
	Establecimiento del objetivo fundamental del proyecto, especificando su alcance. En este capitulo tambien se especifica el producto final y la organizacion que va a tener el proyecto asi como las tareas a realizar y la metodologia utilizada para ello. 
	\item \textbf{Especificacion de requisitos}\smallbreak
	Especificacion de los requisitos para cada elemento del proyecto.
	\item \textbf{Tecnologias utilizadas} \smallbreak
	Definicion de las tecnologias utilizadas, diferentes ejemplos de uso y razon de uso de la tecnologia.	
	\item \textbf{Especificacion del diseño} \smallbreak
	En este capitulo se especifica el diseño de los diferentes elementos del proyecto, asi como las herramientas utilizadas para ello.
	\item \textbf{Consideraciones sobre la implementación} \smallbreak
	Consideraciones a tener en cuenta sobre la implementacion de las diferentes funcionalidades y elementos del sistema.
	\item \textbf{Plan de pruebas} \smallbreak
	Especificacion de las pruebas realizadas para el correcto funcionamiento del sistema.
	\item \textbf{Manual de usuario} \smallbreak
	Desarrollo de los manuales de usuario para el correcto uso del sistema.
	\item \textbf{Incidencia} \smallbreak
	Especificacion de las incidencias ocurridas durante el desarrollo del proyecto y soluciones.
	\item \textbf{Conclusiones y lineas futuras} \smallbreak
	Conclusiones extraidas del proyecto y posibles lineas futuras para aumentar la calidad del proyecto.
\end{itemize}

\section{Motivación}
Alboan ingreso el año 2015 nueve millones de euros. Este dinero proviene de donaciones tanto privadas como públicas. Las donaciones o ayudas públicas dependen del gobierno del año en la que las recibe, por lo que estas no tienen una solución posible ya que no dependen de nadie más que del gobierno. En cambio, las donaciones privadas, provenientes de personas o instituciones no públicas. Gracias a todas estas donaciones Alboan tiene 200 proyectos activos en 18 países diferentes, esto hace que mantener el dinero que la ONG invierte en cada uno de ellos sea crucial año tras año. \\

En el ejercicio de 2014 la financiación privada crece gracias a los legados solidarios\footnote{Son las herencias que ciertas personas dejan a entidades solidarias con el fin de construir un mejor mundo.} que la gente deja a la organización, también son muy importantes las cuotas de los socios que representan un 28\% de las aportaciones privadas. Estas donaciones son importantes para que Alboan cada vez pueda llegar a más gente y pueda crear más proyectos necesarios.\\

Alboan descubre entonces la necesidad de crear un nuevo canal por el que recibir donaciones y hacer que sus proyectos sean mas visibles de cara a la gente que no conoce tanto a Alboan. Entonces es cuando comienzan a gestarse los objetivos y requerimientos que la Colmena deberá cumplir.\\


\figuraSinMarco{0.6}{imgs/cuentas.png}{Ejercicio de cuentas de Alboan}{cuentas}{}
