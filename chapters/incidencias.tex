En este capitulo se explicarán las incidencias mas importantes durante el transcurso del proyecto.

\section{Interacción con el cliente}
Durante todo el transcurso del proyecto la interacción con la ONG Alboan ha sido muy habitual. El cruce de emails entre la persona al cargo del desarrollo tecnológico y el equipo de desarrollo ha sido muy intenso durante varios meses. La disposición y la interacción por parte del cliente ha sido buena\\

El problema radica en los tiempos de respuesta de algunas de las peticiones que el equipo de desarrollo realizaba. Durante un tiempo el equipo de desarrollo tuvo que ir alternando entre las tareas ya que los materiales necesarios para realizar algunas de las tareas no llegaron hasta pasados unos dias.\\

Para paliar este problema se decidió unificar todas las peticiones durante una de las reuniones que se tuvieron con los responsables de la ONG. En su intento de mejorar la situacion ellos se comprometieron a responder a las peticiones que el equipo realizaba lo mas rapidamente posible y asi fue.

\section{Implementacion del sistema de visualizacion de datos}

El sistema de visualizacion estaba diseñado graficos que dependian de la API para funcionar. La API proveia los JSON necesarios para que el sistema funcionase correctamente por lo que antes de ofrecerselos tenia que formatearlos.\\

El problema fue que uno de los JSON que tenia que crear era demasiado complejo y presentaba mucha codificacion, comparandolo con otras opciones. Es por lo tanto que se opta por crear a mano un \textit{CSV} para mostrar dichos datos.