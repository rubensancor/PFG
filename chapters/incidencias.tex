En este capítulo se explicarán las incidencias más importantes durante el transcurso del proyecto.

\section{Interacción con el cliente}
Durante todo el transcurso del proyecto la interacción con la ONG Alboan ha sido muy habitual. El cruce de emails entre la persona al cargo del desarrollo tecnológico y el equipo de desarrollo ha sido muy intenso durante varios meses. La disposición y la interacción por parte del cliente ha sido buena\\

El problema radica en los tiempos de respuesta de algunas de las peticiones que el equipo de desarrollo realizaba. Durante un tiempo el equipo de desarrollo tuvo que ir alternando entre las tareas ya que los materiales necesarios para realizar algunas de las tareas no llegaron hasta pasados unos días.\\

Para paliar este problema se decidió unificar todas las peticiones durante una de las reuniones que se tuvieron con los responsables de la ONG. En su intento de mejorar la situación ellos se comprometieron a responder a las peticiones que el equipo realizaba lo más rápidamente posible y así fue.

\section{Implementación del sistema de visualización de datos}

El sistema de visualización estaba diseñado gráficos que dependían de la API para funcionar. La API proveía los JSON necesarios para que el sistema funcionase correctamente por lo que antes de ofrecérselos tenía que formatearlos.\\

El problema fue que uno de los JSON que tenía que crear era demasiado complejo y presentaba mucha codificación, comparándolo con otras opciones. Es por lo tanto que se opta por crear a mano un \textit{CSV} para mostrar dichos datos.
