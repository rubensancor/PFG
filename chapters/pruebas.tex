Durante el desarrollo del proyecto se han ido realizando diferentes pruebas que han permitido que el proyecto avance de manera regular, en caso de haber algun error se ha planificado su correccion para continuar con el correcto desarrollo del proyecto. Durante este capítulo se explicaran algunas de las pruebas realizadas.

\section{Pruebas del servidor}
Las pruebas del servidor se han realizado mediante una clase que Node.js lleva implementada internamente, la clase \textit{Console}. Esta clase nos permite loguear todos los eventos o mensajes de la aplicacion. Su funcionamiento es similar al famoso framework Log4J. La clase \textit{Console} permite mostrar el mensaje que aparece por consola trantandolo de manera diferente dependiendo de si es un mensaje de error, de log, de alerta...\\

Por otra parte las pruebas unitarias se han realizado con un framework llamado Mocha que permite simular peticiones HTTP e incluso testear funciones asíncronas mediante asertos, como es el caso de las consultas a la base de datos.

\section{Pruebas a la base de datos}
Las pruebas a la base de datos se han hecho en materia de velocidad de peticiones y almacenamiento de los datos ya que al ser una base de datos noSQL no podemos cometer el error de añadir datos que no correspondan a las tablas configuradas.\\

Para estas pruebas se ha creado un Script que genera donaciones de manera automatizada y posteriormente se han hecho peticiones que se pueden dar con el sistema en fucnionamiento. Con esto se ha medido el rendimiento de la base de datos.

\section{Pruebas de la página web y el widget}
Las pruebas de la parte visual del 

\section{Pruebas del sistema de visualización de datos}