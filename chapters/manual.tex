En este capítulo se muestra los manuales para utilizar el sistema paso a paso. Los manuales explicados en el capítulo son los siguientes:

\begin{itemize}
	\item \textbf{Manual del donante:} en este manual se incluyen todos los pasos que una persona debe hacer desde que realiza la donación hasta que recibe el certificado de donación.
	\item \textbf{Manual del comercio online:} en este manual se incluyen todos los pasos que una empresa debe seguir para añadir el widget solidario a su comercio online.
\end{itemize}

\section{Manual del donante}
Partimos desde el punto en el que el usuario se encuentra en un comercio online y se dispone a comprar lo que el desee, en este manual se ha utilizado la tienda de Alboan y el usuario está comprando tarjetas de navidad.


Una vez añadido el euro solidario a su carrito puede finalizar su compra. Una vez finalizada la compra el widget se encargará de enviar un mail a la cuenta de correo que tienes en la página web donde realizaste la compra. \\

Una vez tengamos ese número en el mail nos acercamos a la página web de la Colmena. Y seleccionamos la opción \textit{Quiero recibir un certificado} o bajamos por la página web hasta llegar a esa sección.\\

\figuraSinMarco{1}{imgs/principal.png}{Primera vista al entrar a la web}{principal}{}

Clicamos en el botón donde pone \textit{Obtener certificado} y se nos desplegará una ventana nueva en la que podremos introducir nuestro número de donación. En el caso de que nuestro número haya sido usado o de que lo introduzcamos incorrectamente nos aparecerá un mensaje de aviso que nos indicará el error. \\

\figuraSinMarco{0.8}{imgs/numcertificado.PNG}{Campo para introducir el número de donación}{numcertificado}{}

Una vez introducido nuestro número de donación correctamente la web nos dirigirá a una nueva página en la que se nos mostrará un formulario. Dentro del formulario debemos introducir los datos fiscales para crear el certificado de donación. El formulario comprueba que los datos que hayamos introducido sean válidos, en este caso el único dato que tiene una restricción es el email. \\

\figuraSinMarco{1}{imgs/formularioCertificado.PNG}{Formulario para obtener certificado con fallo en el email}{formulario}{}

Finalmente, si hemos introducido nuestros datos correctamente nos llegará al correo indicado un mensaje en el que ira adjunto nuestro certificado de donación. El certificado de donación se puede utilizar para fines fiscales.\\

Dentro de la página web también se puede obtener información sobre los diferentes proyectos que la ONG tiene abiertos en estos momentos. Para ver estos proyectos podemos navegar hasta la sección donde pone \textit{Proyectos} o clicar en la opción de \textit{Quiero conocer proyectos} que ofrece el menú de la parte superior. Una vez en esta sección podemos clicar encima de cualquiera de los proyectos que se nos ofrecen, entonces se desplegará una nueva ventana en la que se nos ofrecerá más información sobre el proyecto.

\figuraSinMarco{1}{imgs/mujeresProyecto.PNG}{Sección de una ventana con información de un proyecto apoyado por la Colmena}{mujeres}{}

\section{Manual del comercio online}
En el caso de ser una empresa que quiere implementar el widget en su comercio online se tendrían que seguir estos pasos. En primer lugar habría que acceder a la página web de la colmena y acceder a la sección donde pone \textit{Creación} o clicar en la opción donde aparece \textit{Quiero crear un widget} del menú superior.\\

Una vez aquí clicamos en el botón donde aparece \textit{Crear widget} y nos aparecerá una ventana con un asistente en la que podremos crear el widget a nuestra elección. Esto es solo la demo para saber cómo se pueden diseñar los widgets. En el caso de querer crear un widget real deberás ponerte en contacto con el equipo de la Colmena.\\

\figuraSinMarco{1}{imgs/wizard1.PNG}{Demo del asistente para la creación de nuevos widgets}{demowizard}{}

Bajamos hasta la última sección de la página web o clicamos en la opción donde pone \textit{Quiero contactar con vosotros}. Una vez en esta sección podemos rellenar el formulario indicando que queremos crear un nuevo widget para nuestra página web, es entonces cuando el equipo de la Colmena se pondrá en contacto con nosotros y nos enviara una ruta que nos llevará al asistente de creación de widgets.\\

\figuraSinMarco{1}{imgs/contacto.PNG}{Formulario de contacto con el equipo de la Colmena}{contacto}{}

\figuraSinMarco{0.8}{imgs/mensaje.PNG}{Mail recibido por el equipo de la Colmena}{mail}{}

Por último, accedemos a la página web que se nos suministra por correo electrónico y volvemos a estar delante del asistente que nos permite crear el widget solidario para nuestra página web. Al saber cómo funciona podemos diseñar ágilmente el widget.

\figuraSinMarco{1}{imgs/wizard2.PNG}{Asistente de creación para nuevos widgets}{wizard2}{}

Finalmente, ya tenemos nuestro widget creado y lo podremos implementar en nuestra página web mediante el manual que la Colmena nos ofrece. Este manual tiene dos partes, la de la implementación en el comercio online y la que tiene que realizar la Colmena en su servidor online. El manual está ubicado en los anexos.
