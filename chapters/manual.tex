En este capitulo se muestra los manuales para utilizar el sistema paso a paso. Los manuales explicados en el capitulo son los siguientes:

\begin{itemize}
	\item \textbf{Manual del donante:} en este manual se incluyen todos los pasos que una persona debe hacer desde que realiza la donación hasta que recibe el certificado de donación.
	\item \textbf{Manual del comercio online:} en este manual se incluyen todos los pasos que una empresa debe seguir para añadir el widget solidario a su comercio online.
\end{itemize}

\section{Manual del donante}
Partimos desde el punto en el que el usuario se encuentra en un comercio online y se dispone a comprar lo que el desee, en este manual se ha utilizado la tienda de Alboan y el usuario esta comprando tarjetas de navidad. 


Una vez añadido el euro solidario a su carrito puede finalizar su compra. Una vez finalizada la compra el widget se encargará de enviar un mail a la cuenta de correo que tienes en la página web donde realizaste la compra. \\

FOTO TIENDA CON WIDGET

Una vez tengamos ese numero en el mail nos acercamos a la página web de la Colmena. Y seleccionamos la opción \textit{Quiero recibir un certificado} o bajamos por la página web hasta llegar a esa sección.\\

\figuraSinMarco{1}{imgs/principal.png}{Primera vista al entrar a la web}{principal}{}

Clicamos en el boton donde pone \textit{Obtener certificado} y se nos desplegará una ventana nueva en la que podremos introducir nuestro numero de donacion. En el caso de que nuestro numero haya sido usado o de que lo introduzcamos incorrectamente nos aparecerá un mensaje de aviso que nos indicará el error. \\

\figuraSinMarco{0.8}{imgs/numcertificado.PNG}{Campo para introducir el numero de donación}{numcertificado}{}

Una vez introducido nuestro numero de donacion correctamente la web nos dirigira a una nueva pagina en la que se nos mostrara un formulario. Dentro del formulario debemos introducir los datos fiscales para crear el certificado de donación. El formulario comprueba que los datos que hayamos introducido sean validos, en este caso el unico dato que tiene una restriccion es el email. \\

\figuraSinMarco{1}{imgs/formularioCertificado.PNG}{Formulario para obtener certificado con fallo en el email}{formulario}{}

Finalmente si hemos introducido nuestros datos correctamente nos llegará al correo indicado un mensaje en el que ira adjunto nuestro certificado de donación. El certificado de donacion se puede utilizar para fines fiscales. El certificado esta añadido en el capitulo de anexos.\\

Dentro de la página web también se puede obtener información sobre los diferentes proyectos que la ONG tiene abiertos en estos momentos. Para ver estos proyectos podemos navegar hasta la seccion donde pone \textit{Proyectos} o clicar en la opcion de \textit{Quiero conocer proyectos} que ofrece el menu de la parte superior. Una vez en esta seccion podemos clicar encima de cualquiera de los proyectos que se nos ofrecen, entonces se desplegará una nueva ventana en la que se nos ofrecerá mas informacion sobre el proyecto.

\figuraSinMarco{1}{imgs/mujeresProyecto.PNG}{Seccion de una ventana con informacion de un proyecto apoyado por la Colmena}{mujeres}{}

\section{Manual del comercio online}
En el caso de ser una empresa que quiere implementar el widget en su comercio online se tendrian que seguir estos pasos. En primer lugar habria que acceder a la página web de la colmena y acceder a la seccion donde pone \textit{Creación} o clicar en la opcion donde aparece \textit{Quiero crear un widget} del menu superior.\\

Una vez aqui clicamos en el boton donde aparece \textit{Crear widget} y nos aparecerá una ventana con un asistente en la que podremos crear el widget a nuestra eleccion. Esto es solo la demo para saber como se pueden diseñar los widgets. En el caso de querer crear un widget real deberás ponerte en contacto con el equipo de la Colmena.\\

\figuraSinMarco{1}{imgs/wizard1.PNG}{Demo del asistente para la creacion de nuevos widgets}{demowizard}{}

Bajamos hasta la ultima seccion de la página web o clicamos en la opcion donde pone \textit{Quiero contactar con vosotros}. Una vez en esta seccion podemos rellenar el formulario indicando que queremos crear un nuevo widget para nuestra página web, es entonces cuando el equipo de la Colmena se pondra en contacto con nosotros y nos enviara una ruta que nos llevará al asistente de creacion de widgets.\\

\figuraSinMarco{1}{imgs/contacto.PNG}{Formulario de contacto con el equipo de la Colmena}{contacto}{}

\figuraSinMarco{0.8}{imgs/mensaje.PNG}{Mail recibido por el equipo de la Colmena}{mail}{}

Por ultimo accedemos a la página web que se nos suministra por correo electronico y volvemos a estar delante del asistente que nos permite crear el el widget solidario para nuestra página web. Al saber como funciona podemos diseñar agilmente el widget. 

\figuraSinMarco{1}{imgs/wizard2.PNG}{Asistente de creación para nuevos widgets}{wizard2}{}

Finalmente ya tenemos nuestro widget creado y lo podremos implementar en nuestra página web mediante el manual que la Colmena nos ofrece. Este manual tiene dos partes, la de la implementación en el comercio online y la que tiene que realizar la Colmena en su servidor online. El manual esta ubicado en los anexos.