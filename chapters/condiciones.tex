El desarrollo tendrá lugar en Deusto Tech. Aquí es donde tendrá su lugar de trabajo el equipo. El proyecto se desarrollará en el transcurso del año 2016-2017.DeustoTech y Alboan ofrecerán todos los medios necesarios para el buen desarrollo del proyecto.

\section{Instalaciones}
En este apartado se listará el software y hardware utilizado para desarrollar el proyecto.

\subsection{Hardware}
Este es el hardware que se utilizará para desarrollar el proyecto:
\begin{itemize}
	\item Lenovo ThinkPad X220
	\item Monitor Acer AL1714 (Segundo monitor)
\end{itemize}

\subsection{Software}
Este es el software que se utilizará para desarrollar el proyecto. Este sera explicado mas adelante:
\begin{itemize}
	\item Atom
	\item MongoDB
	\item Git
	\item Brackets
\end{itemize}

\subsubsection{Atom}
Atom es un editor de código fuente abierto para macOS, Linux, y Windows con soporte para plug-ins escrito en Node.js, Incrustando Git Control, desarrollado por GitHub. Atom es una aplicación de escritorio construida utilizando tecnologías web. La mayor parte de los paquetes tienen licencias de software libre y es construido y mantenido por su comunidad. Atom está basado en Electrón (Anteriormente conocido como Atom Shell), Un framework que permite aplicaciones de escritorio multiplataforma usando Chromium y Node.js. Está escrito en CoffeeScript y Less. También puede ser utilizado como un entorno de desarrollo integrado (IDE). Atom libero su beta en la versión 1.0, encima junio 25, 2015. Sus desarrolladores lo llaman un "Editor de textos hackable para el siglo XXI".


\subsubsection{MongoDB}
MongoDB es un sistema de base de datos NoSQL orientado a documentos, desarrollado bajo el concepto de código abierto. MongoDB forma parte de la nueva familia de sistemas de base de datos NoSQL. En lugar de guardar los datos en tablas como se hace en las bases de datos relacionales, MongoDB guarda estructuras de datos en documentos similares a JSON con un esquema dinámico (MongoDB utiliza una especificación llamada BSON), haciendo que la integración de los datos en ciertas aplicaciones sea más fácil y rápida.

\subsubsection{Git}
Git es un software de control de versiones diseñado por Linus Torvalds, pensando en la eficiencia y la confiabilidad del mantenimiento de versiones de aplicaciones cuando éstas tienen un gran número de archivos de código fuente. Al principio, Git se pensó como un motor de bajo nivel sobre el cual otros pudieran escribir la interfaz de usuario o front end como Cogito o StGIT. Sin embargo, Git se ha convertido desde entonces en un sistema de control de versiones con funcionalidad plena. Hay algunos proyectos de mucha relevancia que ya usan Git, en particular, el grupo de programación del núcleo Linux.


\subsubsection{Brackets}
Brackets es un editor de codigo fuente abierto escrito en HTML, CSS y JavaScript enfocado en el diseño web. Fue creado por Adobe Systems, bajo licencia del MIT y actualmente mantenido por GitHub. Brackets está disponible para Mac, Windows y Linux.




\section{Equipo de proyecto}
En este apartado se describirá al equipo del proyecto desde dos perspectivas diferentes. En la sección de Interacción con el cliente se describirá como deberá ser el contacto con el cliente poniendo en práctica las buenas prácticas de la metodología Scrum. En el caso de los perfiles profesionales se definirán los perfiles que deberá haber para desarrollar el proyecto y cómo será el organigrama del proyecto.

\subsection{Interacción con el cliente}
El esquema organizativo

\begin{itemize}
	\item \textbf{Product Owner:} Este rol será representado por una persona de Alboan. Sus principales funciones son las siguientes:
		\subitem Ser el representante de todas las personas interesadas en los resultados del proyecto
		\subitem Definir los objetivos del producto o proyecto.
		\subitem Colaborar con el equipo para planificar, revisar y dar detalle a los objetivos de cada iteración.
	\item \textbf{Scrum master:} No se debe confundir con el jefe de proyecto. Sus principales tareas son:
		\subitem Facilitar las reuniones de Scrum
		\subitem Proteger y aislar al equipo de interrupciones externas
		\subitem Quitar los impedimentos que el equipo tiene en su camino
	\item \textbf{Equipo de desarrollo:} Compuesto por los profesionales que desarrollarán el proyecto. Su principal función es desarrollar el proyecto y estas son otras de sus funciones:
		\subitem Equipo auto organizado.
		\subitem Equipo multidisciplinar, capacidad de desarrollar diferentes tareas.
		\subitem Equipo estable durante el proyecto y establecidos en la misma localización física.

\end{itemize}

En la figura 5.1 se puede ver cómo será la interlocución entre los miembros del equipo. El Scrum master, representado por Rubén Sánchez, será el encargado de interaccionar con el Product Owner, que estará representado por una persona de Alboan. Por último, el equipo de desarrollo incluirá a las personas que desarrollen el proyecto, las cuales podremos ver más adelante.

\figura{0.6}{imgs/equiposcrum.png}{Gráfico de la interacción en el proyecto}{equiposcrum}{}

\subsection{Perfiles profesionales}

En este apartado se describirá al equipo de proyecto que será el encargado de desarrollar el producto y los diferentes perfiles necesarios para ello.

\begin{itemize}
	\item \textbf{Jefe de proyecto:} Es el encargado de controlar que el proyecto se desarrolle correctamente y de garantizar los tiempos de desarrollo y la calidad del mismo. También se encargará de los aspectos de gestión del proyecto.
	\item \textbf{Diseñador:} Es el encargado de diseñar la interfaz de la página web y del widget.
	\item \textbf{Analista de datos:} Es el encargado de analizar los datos que se extraen del proyecto y de crear informes interactivos y conclusiones de ellos.
	\item \textbf{Arquitecto web:} Es el encargado de diseñar y desarrollar la arquitectura principal del proyecto. También se encarga de configurar las herramientas que vayan a utilizar los demás para asegurar su correcto funcionamiento.
	\item \textbf{Programador:} Es el encargado de desarrollar los métodos que el arquitecto haya establecido.
\end{itemize}

En la figura 5.2 se ve la organización en el equipo de proyecto. Alboan está como cliente externo a Deusto Tech. Dentro de Deusto Tech tenemos a Pablo García, el cual será el supervisor del proyecto desde la empresa, y al equipo de la Colmena. Dentro del equipo de la Colmena contamos con dos personas, el jefe del proyecto, que hará las veces de Diseñador, Analista de datos y Arquitecto Web y, por otra parte, el programador.

\figura{0.8}{imgs/Equipo.png}{Gráfico del equipo de proyecto}{equipo}{}

\subsection{Procedimientos de seguimiento y control}

El seguimiento del proyecto se hará desde el tablón de Trello. En el tablón se registran los cambios que se generan y se generan notificaciones para las personas involucradas en la tarea que ha sufrido cambios. Finalmente se puede sacar un reporte de los cambios en las tareas.\\

En cuanto a reuniones de seguimiento del proyecto, tendremos una reunión cada viernes en la que el equipo de proyecto hablará con el supervisor de Deusto Tech. En esa reunión se podrán ver los avances y problemas que hay cada semana y se pensarán diferentes soluciones para ellos. Estas reuniones también servirían por si se necesitará algo de Deusto Tech.
