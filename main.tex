\documentclass{memoriaPFC}
\fontfamily{cmss}\selectfont



%
% DATOS DEL DOCUMENTO
%

\title{Colmena: diseño e implementación de un widget web para micro donaciones, con página web de soporte, almacenamiento de datos y su posterior visualización}

%\autores{Nombre1 Apellidos1}{Nombre2 Apellidos2} Hasta 5 autores
\autores{Rubén Sánchez Corcuera}{}{}{}{}

%El director del proyecto
\director{Pablo García Bringas}
%\directora{Ursula K. Le Guin}

% iaei ii ioi itel (ii es un pdf que contiene los logos para el grado de informatica, el unico actualizado)
\titulacion{ii}

% mayo || septiembre (en minúsculas!!)
\date{mayo de 2017}

\resumen{%
 El proyecto consta de varios módulos. La base de todo es un servidor implementado en node.js. En él se aloja la página web, los widgets de las diferentes empresas que han decidido implantarlo en sus páginas web y una restful API que administra el enrutamiento de la web y permite varias funciones sobre la base de datos.\\
 
 La página web sirve de apoyo al widget, es donde se muestra la información general del widget, los diferentes proyectos a los que apoyar, un wizard con el que poder crear tu propio widget y la plataforma para poder conseguir el certificado de donación.\\
 
 La base de datos noSQL aloja los datos de las diferentes donaciones para posteriormente poder crear el certificado de donación y eventualmente generar estadísticas sobre las cantidades, fechas y lugares en los que más donaciones se están obteniendo.\\
 
 Por último, el widget. Está pensado para que las empresas que tengan portales de compra online lo incrusten durante algunas de sus fases compra del cliente con el fin de que este haga una pequeña donación a la causa que las empresas han “apadrinado”. El widget es personalizable por las empresas que lo quieran implantar en su web, así conseguir una mayor integración con ellas. Entre sus opciones de personalización destaca la posibilidad de elegir entre una cantidad fija o variable de donación.
  
 
}
\descriptores{solidaridad, widget, RESTful, NodeJS, MongoDB}


%
% COMIENZO DEL DOCUMENTO
%
\begin{document}

% Portada, resumen, indices
\frontmatter
\hacerportada
\hacerresumen
\tableofcontents
\listoffigures % Opcional
\listoftables % Opcional
%\lstlistoflistings % Opcional

% Contenidos
\mainmatter
\chapter{Introducción}

\section{Presentación del documento}

El presente documento recoge proyecto desarrollado por el alumno Rubén Sánchez para la ONG ALBOAN. En el proyecto consta de todo el sistema necesario para crear un sistema de micro donaciones mediante un widget en cualquier tienda online. El sistema consta de: una página web, una base de datos, el widget y un servidor que da soporte a todo. Además, el sistema cuenta con una funcionalidad añadida por la que se puede crear gráficas de visualización.\\

El proyecto se desarrolla para la ONG Alboan la cual al ver el cambio que se da en la sociedad en materia de solidaridad y que las grandes donaciones han sufrido una gran caída decide que la manera de recoger las donaciones tiene que cambiar. Alboan, al explorar las oportunidades y ver que la única opción disponible cobra a las ONGs un porcentaje de las donaciones decide crear un sistema de micro donaciones gratuito que cualquier empresa pueda añadir a su tienda online y así fomentar las donaciones a pequeña escala, consiguiendo así que el 100\% del dinero vaya al destino final.\\

Los principales capítulos del documento son los siguientes:

\begin{itemize}
	\item \textbf{Definición del proyecto}\smallbreak
	Establecimiento del objetivo fundamental del proyecto, especificando su alcance.
	\item \textbf{Producto final}\smallbreak
	Especificación de los elementos que componen el proyecto Colmena.
	\item \textbf{Organización} \smallbreak
	Definición del equipo de trabajo que desarrollará el proyecto y los perfiles profesionales que formaran parte de este. También incluye la estructura organizativa y el sistema utilizado para gestionar el proyecto.
	\item \textbf{Condiciones de ejecución} \smallbreak
	Definición del entorno de trabajo y del hardware y software a utilizar y de la metodología que se utilizará para hacer las modificaciones o mejoras que alteren el planteamiento inicial en el proyecto.
	\item \textbf{Planificación} \smallbreak
	Estimación de la duración de las tareas durante el transcurso del proyecto, así como su planificación en el tiempo.
	\item \textbf{Valoración económica} \smallbreak
	Determinación del valor correspondiente a este proyecto, las horas de desarrollo y las herramientas y elementos utilizados.
	\item 
\end{itemize}

\section{Motivación}
Alboan ingreso el año 2015 nueve millones de euros. Este dinero proviene de donaciones tanto privadas como públicas. Las donaciones o ayudas públicas dependen del gobierno del año en la que las recibe, por lo que estas no tienen una solución posible ya que no dependen de nadie más que del gobierno. En cambio, las donaciones privadas, provenientes de personas o instituciones no públicas. Gracias a todas estas donaciones Alboan tiene 200 proyectos activos en 18 países diferentes, esto hace que mantener el dinero que la ONG invierte en cada uno de ellos sea crucial año tras año. \\

En el ejercicio de 2014 la financiación privada crece gracias a los legados solidarios\footnote{Son las herencias que ciertas personas dejan a entidades solidarias con el fin de construir un mejor mundo.} que la gente deja a la organización, también son muy importantes las cuotas de los socios que representan un 28\% de las aportaciones privadas. Estas donaciones son importantes para que Alboan cada vez pueda llegar a más gente y pueda crear más proyectos necesarios.\\

Alboan descubre entonces la necesidad de crear un nuevo canal por el que recibir donaciones y hacer que sus proyectos sean mas visibles de cara a la gente que no conoce tanto a Alboan. Entonces es cuando comienzan a gestarse los objetivos y requerimientos que la Colmena deberá cumplir.\\


\figuraSinMarco{0.6}{imgs/cuentas.png}{Ejercicio de cuentas de Alboan}{cuentas}{}


\chapter{Objetivos del proyecto}

\smallbreak %borrar
En este capítulo hablaremos sobre el beneficiario del proyecto, en este caso es la ONG ALBOAN.

\section{Alboan}

La actividad de Alboan comenzó en 1994, asumiendo las iniciativas de voluntariado
internacional que ya estaban en marcha, pero fue en 1996 cuando se configuró
jurídicamente bajo la figura de Fundación. Es la ONG promovida por la Compañía de Jesús en la Provincia de Loyola (País Vasco y Navarra). El nombre, en euskera, Alboan, quiere reflejar el arraigo a la cultura de la tierra vasca y el espíritu de la entidad: estar al lado de las personas más excluidas, junto a organizaciones y centros educativos.\\

Su logo(Figura \ref{logo}) visibiliza  el rol de bisagra y puente que quería jugar la institución para poner en relación dos mundos que en realidad son uno solo. Su misión: ser plataforma de encuentro de personas y organizaciones de aquí y allá que quisieran comprometerse en la construcción de un mundo mejor.\\

Hoy Alboan cuenta con 7000 personas entre contratados, voluntarios y entidades que le permiten apoyar y acompañar a más de 200 proyectos llevados a cabo por 105 organizaciones
aliadas que impactan directamente en las vidas de más de 600.000 personas en todo
el mundo.

\figura{0.5}{imgs/alboan.jpg}{Logo de Alboan}{logo}{}

\subsection{Misión y Visión}

\subsubsection{Misión}

Trabaja por la construcción de una ciudadanía global que denuncie las injusticias que provocan desigualdad en el mundo, construya una cultura que promueva el bien común y transforme las estructuras generadoras de pobreza a nivel local y global. Para lograrlo, se une en red con personas y grupos de todo el mundo.\\

La colaboración de Alboan se centra en 4 temáticas principales en las que enmarca todos sus proyectos y labores de ayuda: Educación de calidad, Desarrollo económico-productivo sostenible y equitativo, Acción humanitaria en crisis recurrentes y Democracia a favor de las personas excluidas.\\

Todas las actividades que realiza la ONG incluyen 3 ejes transversales que otorgan a los proyectos el carisma que Alboan quiere dar:
\begin{itemize}
	\item La espiritualidad como dimensión en el horizonte de desarrollo humano.
	\item El reconocimiento de las desigualdades entre mujeres y hombres y el compromiso con la equidad de género.
	\item La participación ciudadana para la incidencia social y política.
\end{itemize}

\subsubsection{Visión}

Lograr una Alboan que contenga las siguientes características:
\begin{itemize}
	\item \textbf{Enraizada} en el nuevo proyecto unificado de la Compañía de Jesús a través de sus plataformas locales y territorial.
	\item \textbf{Querida} por las organizaciones y la base social con las que se alía.
	\item \textbf{Reconocida} por su valor añadido en el acompañamiento a entidades, la
	formación y la construcción de ciudadanía global.
	\item \textbf{Sostenible} gracias a un equipo comprometido y una financiación estable y
	diversificada.
	\item \textbf{Ilusionante}, por sus propuestas y su comunicación esperanzadora.
	\item \textbf{Puente} entre nuestro estilo de vida y las situaciones de frontera de
	deshumanización
\end{itemize}


\subsection{Dedicación}

Alboan tiene unos ámbitos de dedicación marcados en los que

\begin{itemize}
	\item La \textbf{Cooperación al Desarrollo} en África, Asia, Centroamérica y Sudamérica, principalmente con proyectos educativos, de promoción económica y de formación de grupos excluidos para la defensa de sus derechos.

	\item La \textbf{Educación para la solidaridad} en Euskadi y en Navarra, mediante la participación en campañas, la elaboración de materiales educativos, la promoción del comercio justo, impartiendo formaciones y talleres y asesorando a grupos y centros educativos.

	\item La \textbf{Acción Política} participando en redes y elaborando estudios e investigaciones para incidir y mejorar las políticas que afectan al desarrollo tanto locales como internacionales.


\end{itemize}


\chapter{Situación de partida}

\small %Quitar
En este capítulo se habla sobre la necesidad de la cual surge crear este proyecto, la motivación por la cual se decide desarrollar el producto final y los objetivos que se pretender conseguir con el mismo.

\section{Necesidades}





Años atrás Alboan comienza a detectar una bajada en la forma tradicional de hacer donaciones. Es entonces cuando se plantea buscar nuevos métodos para realizar estas donaciones y que el dinero siga fluyendo. En el ejercicio de 2015 la financiación privada fue ligeramente menor que el año anterior y se rompió la tendencia en alza. A parte de tener menos financiación Alboan se da cuenta de que algunos de sus proyectos no son conocidos y que la ayuda para que estos continúen es crucial, por lo que la preocupación hacia estos proyectos crece.\\

Es entonces cuando le surgen a Alboan varias necesidades, la de volver a conseguir financiación privada ya que no pueden sustentarse únicamente de los legados solidarios y las cuotas que pagan sus socios y por otra parte la de hacer llegar a más personas sus proyectos para que de esta manera aumente su sensibilidad con el tema. Para lograr cubrir estas necesidades Alboan intenta pensar en una solución en la que los proyectos de la ONG se den a conocer y que la gente que este interesada en ellos pueda hacer una pequeña aportación, en este momento surge la idea de las pequeñas donaciones.\\


\newpage

\section{Objetivos}

El principal objetivo de este proyecto es crear un widget solidario que se pueda implantar en cualquier tienda online y crear la página web en la que se va a apoyar y se dará a conocer este widget. La página web también será el portal en el que la gente que ha realizado alguna donación pueda recoger su certificado de donación con el fin de presentarlo en la declaración de la renta.\\
Otro objetivo sería el de crear el servidor con la base de datos para alojar los datos de las donaciones realizadas mediante el widget. Este servidor seria accesible para las personas de Alboan que quisieran consultar los datos de las donaciones o ver las gráficas relacionadas con esto.  \\

\section{Requerimientos}

En este apartado listaremos los diferentes requerimientos del proyecto.

\begin{itemize}
	\item Formar a las personas que realizarán el mantenimiento del sistema mediante tutoriales.
	\item Utilizar las últimas tecnologías en diseño web.
	\item Crear una web responsiva para que todos los usuarios puedan acceder a ella y que evite la brecha digital.
	\item Garantizar los tiempos de respuesta del widget para que las empresas que lo usen no vean perjudicados sus comercios online.
	\item Visualización de datos comprensible e intuitiva.
\end{itemize}

\section{Estado del arte}
En este apartado se hace un análisis de las diferentes opciones que hay para cubrir las necesidades citadas anteriormente.
\subsection{Crowdfunding}
“Cooperación colectiva, llevada a cabo por personas que realizan una red para conseguir dinero u otros recursos, se suele utilizar Internet para financiar esfuerzos e iniciativas de otras personas u organizaciones”\cite{crowd}\\

En el crowdfundig hay diferentes tipos de ayudas y de sistemas creados para financiar los proyectos que la gente sube a las plataformas: basado en donaciones (migranodearena, teaming...), basados en recompensas(Goteo, Verkami...) y para préstamos(ECrowd!) entre otros.

\subsubsection{Basado en donaciones}
El usuario realiza una donación para ayudar a un proyecto que le gusta y no recibe ningún beneficio económico en retorno (solo la satisfacción de hacer algo bueno y de apoyar un proyecto que le emociona). Son proyectos solidarios con un impacto social grande.
\begin{itemize}
	\item \textbf{Migranodearena.org:}\smallbreak
	La página web migranodearena es un proyecto creado por la Fundación Privada real dreams. Es una plataforma de crowdfunding solidario, pionera en España, abierta para todas las Entidades No Lucrativas, legalmente constituidas y con sede en España. \\
	En migranodearena se puede liderar un reto y recaudar fondos en grupo a favor de una ONG. Una persona o grupo de personas, lanza un reto en migranodeareana para apoyar una causa solidaria. Comparte el reto con todos sus familiares, amigos, conocidos, clientes, empleados, proveedores y entre todos suman granitos de arena.
	\item \textbf{Teaming:}\smallbreak
	Teaming es una herramienta online para recaudar fondos para causas sociales a través de micro donaciones de \EUR{1} al mes. La filosofía de Teaming se basa en la idea de que con \EUR{1}, nosotros solos no podemos hacer mucho pero si nos unimos, podemos conseguir grandes cosas.\\
	El sistema de funcionamiento de Teaming es muy sencillo, puedes publicar un grupo en el que la gente se apunte y done \EUR{1} al mes o puedes apuntarte a un grupo que otra persona haya publicado para donar tu ese euro. Entre las ideas clave mas destacadas de la plataforma están la de solo donar \EUR{1} al mes, ni más ni menos y la de solo aceptar causas sociales.
\end{itemize}

\subsubsection{Basado en recompensas}
Se trata de aportar dinero a un proyecto y se recibe a cambio una recompensa. Los proyectos pueden consistir en crear y diseñar un nuevo producto o financiar un proyecto cultural (película, festival, musical). Es el modelo de crowdfunding más utilizado y conocido. En cada proyecto se puede elegir entre un rango de recompensas en función de la cantidad de dinero que se aporte.

\begin{itemize}
	\item \textbf{Goteo:} \smallbreak
	A simple vista, Goteo es una plataforma de crowdfunding cívico y colaboración en torno a iniciativas ciudadanas, proyectos sociales, culturales, tecnológicos y educativos. Con réplicas y alianzas en varios países, gracias a su código abierto, además de reconocida y premiada internacionalmente desde 2011. Constituye una herramienta de generación de recursos, gota a gota, para una comunidad de comunidades compuesta por más de 65.000 personas, con un porcentaje de éxito de financiación superior al 70\%.\\
	Pero en realidad Goteo es mucho más que eso. Tras la plataforma existe una fundación sin ánimo de lucro y un equipo multidisciplinar desde el que desarrollan herramientas y servicios de co-creación y financiación colectiva. Con una misión común vinculada siempre a principios de transparencia, progreso y mejora de la sociedad.
\end{itemize}

\subsubsection{Para prestamos}
El usuario realiza una inversión en un proyecto o en una empresa de la plataforma y recibe un retorno económico en forma de intereses. La empresa contrata un préstamo con la plataforma de crowdlending, que lo gestiona, y el usuario recupera el dinero invertido junto a los intereses a lo largo del tiempo según las condiciones pactadas en el préstamo.

\begin{itemize}
	\item \textbf{ECrowd!:} \smallbreak
	ECrowdInvest es una plataforma de crowdlending (crowdfunding en forma de préstamo) para proyectos con impacto social y medio ambiental positivo (lo que venimos a llamar impacto positivo). Es una clara versión del clásico "win-win", donde todos ganan, ya que los buenos proyectos consiguen financiarse, los inversores reciben unos intereses mucho más altos de los que se obtienen con los bancos y siempre se genera un impacto positivo sobre el medio ambiente, por ejemplo, en la financiación colectiva de proyectos que impliquen la reducción de emisiones de dióxido de carbono hacia la atmósfera.
\end{itemize}


\subsection{Micro donaciones}

Las micro donaciones, como su nombre indica, son donaciones muy pequeñas realizadas con el fin de aportar una pequeña cantidad de dinero al proyecto que nos interesa. En este ámbito nos encontramos con Worldcoo una herramienta online con bastante recorrido y reconocimiento.

\begin{itemize}
	\item \textbf{Worldcoo:} \smallbreak
	Worldcoo es una empresa nacida en 2012 que pretende crear un nuevo canal de financiación mediante un widget incrustado en tiendas online. Esto permite que las miles de personas que acceden y compran en las tiendas online puedan donar una pequeña cantidad al proyecto que esa página tiene apadrinado.\\
	Cuentan con un extenso equipo de personas y de embajadores en varios paises del mundo, lo que les da acceso a muchos comercios online. Las empresas contactan con Worldcoo y acuerdan entre los dos la instalación del widget en su tienda online. Este widget esta relacionado con uno de los proyectos de las ONGs a las que Worldcoo ayuda.
\end{itemize}

Esto es exactamente lo que Alboan quiere, pero tiene algunas carencias. La empresa Worldcoo se queda un 8\% de las donaciones realizadas a cada proyecto por lo que esto empieza a disgustar a Alboan, esta empresa tampoco permite ver las personas que están donando a los proyectos por lo que no permite mantener el contacto con esas personas para hacerles conscientes de que su ayuda tiene un resultado. \\

Después de hacer un análisis de todas las opciones y ver los resultados Alboan decide crear su propia plataforma de micro donaciones mediante widgets en tiendas online.


\chapter{Definición del proyecto}

\small %quitar
En este apartado se procede a definir el proyecto, la metodología utilizada para desarrollarlo y el alcance del mismo. Durante todo el desarrollo del proyecto y las decisiones tomadas en el mismo se ha tenido en mente lo escrito en este apartado.

\section{Descripción del producto}

En este apartado se presenta una descripción completa del producto final, el cual consta de 4 partes, que juntas, forman el sistema completo Colmena.\\

El sistema Colmena consta de las 4 partes que se muestran en la Figura 4.1. El servidor central que une las otras 3 partes es el núcleo del proyecto. A este servidor se conectan los otros 3 componentes de la infraestructura completa a las que el servidor otorga funcionalidad. En la página web se permite al usuario informarse sobre los proyectos y sobre el proyecto Colmena en general. También ofrece la expedición de certificados y la creación de nuevos widgets. En la base de datos se almacenan los datos respectivos a las donaciones y a los donantes, en este caso el servidor extrae e introduce datos en ella para poder gestionar los certificados de donación. Por último, el widget, este es el producto final que la mayoría de los usuarios verán. Este se coloca en los comercios online para que las personas puedan hacer una pequeña aportación a los proyectos que la tienda haya apadrinado.

\begin{figure}[h]
	\centering
	\includegraphics[width=0.5\textwidth]{imgs/descripcion.png}
	\caption{Infraestructura de Colmena}
	\label{infraestructura}
\end{figure}

\subsection{Widget}
El widget es la parte principal del proyecto. Es el objeto que las empresas colocaran en sus tiendas online y que permitirá a los clientes hacer una donación. Consiste en un pequeño rectángulo(\ref{widget}) en el que se le ofrece al a persona que está leyéndolo que aporte una pequeña cantidad de dinero, siempre entre 1 y 5 euros, a un proyecto en concreto. \\

El widget es 100\% personalizable. En la web existe un asistente que permite a las empresas personalizar el widget de modo que este encaje bien en sus comercios online. Este asistente permite cambiar las siguientes características del widget:

\begin{itemize}
	\item \textbf{Proyecto:} Permite elegir a que proyecto destinar el dinero entre los proyectos de Alboan.
	\item \textbf{Opciones de donación:} Permite decidir si la donación será estática, de \EUR{1}, o si esta podrá ser variable, entre 1 y 5 euros.
	\item \textbf{Fondo:} Permite elegir entre una foto, única para cada proyecto, o un fondo de color estático mediante un selector de color.
	\item \textbf{Color de la fuente:} Permite elegir el color de la fuente
\end{itemize}

El widget está compuesto de lenguaje HTML5 y CSS3, no tiene ningún plugin externo, lo que permite incrustarlo en cualquier página web sin que esta se vea afectada o haya que hacer alguna modificación en ella. Se ha desarrollado para que sea responsivo.\\

Por último, la inclusión del widget en la tienda esta simplificada al máximo de manera que la tienda no se vea afectada o tenga que realizar labores de reestructuración del código de su página web. Este proceso se hace mediante un script de código en JavaScript que permite añadir, en el espacio reservado para el mismo, el widget.

\begin{figure}[h]
	\centering
	\includegraphics[width=1\textwidth]{imgs/widget.png}
	\caption {Widget Colmena}
	\label{widget}
\end{figure}



\subsection{Pagina web}

La página web de la Colmena cubre las labores de publicidad y de soporte del widget. Esta incluye funcionalidades como recoger el certificado de donación después de haber realizado una, crear nuevos widgets para las empresas que quieran añadirlos a su página web o contacto con el soporte de Alboan.\\

La página esta desarrollada en \textit{HTML5}, \textit{CSS3}(mediante el pre procesador \textit{sass}) y \textit{JavaScript}. A parte de estas tecnologías se ha utilizado el framework web \textit{Bootstrap} y una gran cantidad de plugins que permiten una mejor navegabilidad en la página web, entre ellos destaca \textit{JQuery}. Se han seleccionado estas tecnologías tras una investigación sobre las tecnologías más recomendadas para el tipo de proyecto. Dado que no teníamos ningún requerimiento sobre estas, hemos apostado por las tecnologías que más nos apetecía utilizar, innovando en algunas de las situaciones y siendo más conservadores en otras.\\

La página web permite al usuario consultar cierta información y realizar varias opciones. Estas opciones se explican a continuación:

\begin{itemize}
	\item \textbf{Conocer el proyecto Colmena:} Es el comienzo de la página web, tras la introducción, compuesta por una foto, está localizada este apartado. Aquí se explica el funcionamiento y la filosofía del proyecto Colmena.
	\item \textbf{Conocer proyectos de Alboan:} En este apartado se exponen los diferentes proyectos que dispone la ONG Alboan. Los proyectos están expuestos con el nombre de este y el país en el que actúa. Si hacemos clic en cualquiera de los proyectos una ventana se despliega y ofrece más información sobre este.
	\item \textbf{Crear un widget:} En este apartado de la página web se ofrece la posibilidad de crear un widget personalizado. Al hacer clic en el botón que ofrece esta posibilidad se despliega una ventana con un asistente. En este asistente se ofrecen varias opciones para personalizar el widget final. Este asistente no es el definitivo que las empresas usaran para realizar el widget que quieren que sea colocado en su comercio online.
	\item \textbf{Recibir un certificado:} En este apartado los usuarios pueden utilizar el código de donación que reciben al donar en cualquier tienda que tenga implantada la Colmena. Después de introducir los datos en un formulario, se le enviará el certificado de donación al mail que ha especificado.
	\item \textbf{Contactar con el soporte de Colmena:} Este es el último apartado de la página web. En él se puede enviar un mail al soporte de Colmena. Es el método de interlocución entre las empresas que quieran implantar el widget en sus comercios y el soporte de Colmena.
\end{itemize}
\newpage
\subsection{Base de datos}

La base de datos de la Colmena está desarrollada en la base de datos NoSQL por excelencia, MongoDB. Se ha utilizado este tipo de base de datos por la versatilidad que proporciona a la hora de acceder a los datos y al conectarse a ella. Es una base de datos muy ágil por lo que permite consultar en un tiempo muy reducido las donaciones. Por último, se ha utilizado esta base de datos por la experimentación con la misma, al no ser una base de datos SQL el equipo tuvimos las ganas de probarla y ver su potencial.\\

La base de datos se encarga de almacenar todos los datos relativos a las donaciones mediante los siguientes campos:

\begin{itemize}
	\item \textbf{Importe:} El importe de la donación.
	\item \textbf{Usada:} Un booleano que marca si la donación ha sido canjeada por el certificado o no.
	\item \textbf{Fecha:} La fecha dividida por día, mes y año.
	\subitem Día
	\subitem Mes
	\subitem Año
	\item \textbf{idDonacion:} Un id asignado a cada donación.\\
\end{itemize}

Una vez almacenada la donación esta está disponible para canjearla por un certificado de donación en la página web. Una vez el certificado de donación es expedido, los datos de la persona donante se guardan en la base de datos con el fin de hacerle llegar información sobre los proyectos, si así lo desea, o hacer diferentes estadísticas con las que la organización pueda mejorar o cambiar sus métodos. Estos son los datos que se añaden al archivo de la donación:

\begin{itemize}
	\item DNI/CIF: Número de identificación fiscal de la persona física o jurídica.
	\item Nombre y Apellidos: El nombre y los apellidos de la persona física o nombre de la empresa.
	\item Razón social: Denominación por la cual se conoce colectivamente a la empresa y en caso de ser una persona, introducirá la palabra "Individuo"
	\item Correo electrónico
	\item Dirección
	\item Código Postal
	\item Población
	\item Provincia
\end{itemize}


\subsection{Servidor}


El servidor del proyecto es el núcleo del mismo. En el convergen todas las funcionalidades de la solución. Este está desarrollado en Node.js. Tras una investigación de las posibles tecnologías y tras ver que no había requerimientos en este aspecto decidimos utilizar esta herramienta ya que es innovadora y emergente.\\

La funcionalidad que el servidor ofrece a los demás componentes del proyecto se basa en recabar la información necesaria de la base de datos y ofrecérsela a la página web para que esta la utilice en sus funciones. También recaba la información del widget y se lo envía a la base de datos para que esta la almacene. Por último el servidor también se utiliza como sistema de almacenamiento de los widgets para tenerlos centralizados y poder repararlos rápidamente en caso de error.\\

El servidor se desarrolla de esta manera, como pieza central, por el hecho de unificar todas las funcionalidades que el sistema pueda necesitar en el mismo nodo. Gracias a la flexibilidad de Node.js y a los plugins que se ofrecen en \textit{NPM}, el gestor de paquetes para \textit{JavaScript}. Con estos plugins se ha conseguido unificar todas las funcionalidades que sin ellos habría que haber desarrollado mediante otros métodos, por ejemplo, el envío de mails o el sistema de creación del certificado de donación.

\subsection{Visualización}

La visualización de los datos es la parte final del proyecto. Este módulo permite visualizar mediante una serie de gráficos interactivos los datos de las donaciones. Los gráficos, como ya he dicho anteriormente son interactivos para que los usuarios puedan 'jugar' con ellos e ir relacionando los datos. Estos gráficos están divididos en dos:

\begin{itemize}
	\item \textbf{Gráfico de queso:} En este gráfico interactivo los usuarios pueden ir clusterizando la información en diferentes divisiones. La primera división por ejemplo seria la dividir las donaciones por el proyecto al que han donado y posteriormente se podría ir profundizando en más opciones.
	\item \textbf{Gráfico con mapa:} En este gráfico se puede ver un mapa de calor en el que se pueden ver las zonas desde las que más se ha donado.
\end{itemize}

Posteriormente estos gráficos se pueden analizar para sacar conclusiones de ellos y poder ofrecer esta información de vuelta a los donantes y socios de la ONG. Con esta información los donantes pueden ver lo que la entidad está recibiendo y como es la sociedad que les rodea gracias al mapa de calor. Por último, Alboan podrá analizar esta información y tomar algunas decisiones basándose en ella además de añadirla a los informes que publica.


\newpage

\subsubsection{Investigación sobre las tecnologías a utilizar}
En esta tarea se investigarán las principales tecnologías a utilizar en el proyecto principal. Se buscarán las mejores tecnologías para desarrollar un proyecto web integral. Las principales funciones de las tecnologías serán las siguientes, crear un servidor web, crear una página web, crear un widget responsivo que se comunique con el servidor y crear una base de datos que aloje los datos.

\subsubsection{Redactar el documento sobre las tecnologías a utilizar}
En esta tarea se redactará lo decidido en la tarea anterior. Se explicarán las diferentes tecnologías que se utilizarán y porque se han elegido estas.

\subsubsection{Instalar las herramientas de desarrollo}
En esta tarea se instalarán las herramientas pertinentes para el desarrollo del proyecto y su correspondiente gestión.

\subsubsection{Configurar la herramienta de control de versiones}
En esta tarea se instalará la herramienta seleccionada anteriormente para el control de versiones en online. Ya que en el desarrollo estarán implicadas más de una persona, se utilizará una herramienta de este tipo para controlar más el desarrollo.

\subsubsection{Diseñar la arquitectura del proyecto}
El objetivo de esta tarea será el de crear una arquitectura completa del proyecto que tenga en cuenta todos los elementos de este. Se crearán los diferentes proyectos en las herramientas de programación, con los métodos más generales. También habrá que crear las conexiones entre las diferentes tecnologías.

\subsubsection{Crear el primer diseño de la página web}
En esta tarea se crearán varios mockups para la interfaz de la página web, que luego serán entregados al cliente para que este las revise y decida cual desea.

\subsubsection{Crear el primer diseño del widget}
En esta tarea se crearán varios mockups para la interfaz del widget, que luego serán entregados al cliente para que este las revise y decida cual desea.

\subsubsection{Desarrollar la lógica del servidor}
En esta tarea se desarrollarán todos los métodos y servicios que el servidor tenga que implementar para dar servicio a todos los componentes del proyecto.

\subsubsection{Pruebas servidor}
En esta tarea se deberá probar el servidor ante todo tipo de situaciones para evitar fallos futuros.

\subsubsection{Configurar la base de datos}
En esta tarea se deberá configurar la base de datos para que sea capaz de alojar los datos que el servidor, el widget y la página web.

\subsubsection{Pruebas base de datos}
En esta tarea se deberá probar la base de datos ante todo tipo de situaciones para evitar fallos futuros.

\subsubsection{Desarrollar la lógica de la página web}
En esta tarea se desarrollará el sistema de enrutado y de funcionalidad de la página web.

\subsubsection{Desarrollar la lógica del widget}
En esta tarea se desarrollará la funcionalidad y la conexión con el servidor del widget.

\subsubsection{Consolidar el diseño de la página web}
En esta tarea se establecerá el diseño definitivo de la página web.

\subsubsection{Pruebas de la página web}
En esta tarea se deberá probar la página web ante todo tipo de situaciones para evitar fallos futuros.

\subsubsection{Consolidar el diseño del widget}
En esta tarea se establecerá el diseño definitivo de la página web.

\subsubsection{Pruebas del widget}
En esta tarea se deberá probar el widget ante todo tipo de situaciones para evitar fallos futuros.

\subsubsection{Crear manual de usuario para los técnicos de mantenimiento}
En esta tarea se deberá crear un manual que indique detalladamente como son todos los procesos que hay que realizar para mantener el sistema y crear nuevos widget para las empresas.

\subsubsection{Investigar los frameworks de visualización de datos}
En esta tarea se deberán investigar los frameworks que se utilizarán a posteriori para visualizar los datos extraídos de las donaciones. Los gráficos deberán ser interactivos.

\subsubsection{Configurar el framework de visualización de datos}
En esta tarea se configurará el framework elegido en la tarea anterior.

\subsubsection{Pruebas sobre el framework de visualización de datos}
En esta tarea se deberá probar el framework de visualización de datos ante todo tipo de situaciones para evitar fallos futuros.

\subsubsection{Analizar los datos}
En esta tarea se analizarán los datos extraídos de la base de datos y se cargarán en diagramas para su posterior visualización.

\subsubsection{Extraer las estadísticas de los datos}
En esta tarea se deberán extraer conclusiones de los datos y análisis realizados en la anterior tarea. Se deberá realizar un documento que recoja las conclusiones extraídas del proceso de análisis de los datos.



\chapter{Condiciones de ejecución}

El desarrollo tendrá lugar en Deusto Tech. Aquí es donde tendrá su lugar de trabajo el equipo. El proyecto se desarrollará en el transcurso del año 2016-2017.DeustoTech y Alboan ofrecerán todos los medios necesarios para el buen desarrollo del proyecto.

\section{Instalaciones}
En este apartado se listará el software y hardware utilizado para desarrollar el proyecto.

\subsection{Hardware}
Este es el hardware que se utilizará para desarrollar el proyecto:
\begin{itemize}
	\item Lenovo ThinkPad X220
	\item Monitor Acer AL1714 (Segundo monitor)
\end{itemize}

\subsection{Software}
Este es el software que se utilizará para desarrollar el proyecto. Este sera explicado mas adelante:
\begin{itemize}
	\item Atom
	\item MongoDB
	\item Git
	\item Brackets
\end{itemize}

\subsubsection{Atom}
Atom es un editor de código fuente abierto para macOS, Linux, y Windows con soporte para plug-ins escrito en Node.js, Incrustando Git Control, desarrollado por GitHub. Atom es una aplicación de escritorio construida utilizando tecnologías web. La mayor parte de los paquetes tienen licencias de software libre y es construido y mantenido por su comunidad. Atom está basado en Electrón (Anteriormente conocido como Atom Shell), Un framework que permite aplicaciones de escritorio multiplataforma usando Chromium y Node.js. Está escrito en CoffeeScript y Less. También puede ser utilizado como un entorno de desarrollo integrado (IDE). Atom libero su beta en la versión 1.0, encima junio 25, 2015. Sus desarrolladores lo llaman un "Editor de textos hackable para el siglo XXI".


\subsubsection{MongoDB}
MongoDB es un sistema de base de datos NoSQL orientado a documentos, desarrollado bajo el concepto de código abierto. MongoDB forma parte de la nueva familia de sistemas de base de datos NoSQL. En lugar de guardar los datos en tablas como se hace en las bases de datos relacionales, MongoDB guarda estructuras de datos en documentos similares a JSON con un esquema dinámico (MongoDB utiliza una especificación llamada BSON), haciendo que la integración de los datos en ciertas aplicaciones sea más fácil y rápida.

\subsubsection{Git}
Git es un software de control de versiones diseñado por Linus Torvalds, pensando en la eficiencia y la confiabilidad del mantenimiento de versiones de aplicaciones cuando éstas tienen un gran número de archivos de código fuente. Al principio, Git se pensó como un motor de bajo nivel sobre el cual otros pudieran escribir la interfaz de usuario o front end como Cogito o StGIT. Sin embargo, Git se ha convertido desde entonces en un sistema de control de versiones con funcionalidad plena. Hay algunos proyectos de mucha relevancia que ya usan Git, en particular, el grupo de programación del núcleo Linux.


\subsubsection{Brackets}
Brackets es un editor de codigo fuente abierto escrito en HTML, CSS y JavaScript enfocado en el diseño web. Fue creado por Adobe Systems, bajo licencia del MIT y actualmente mantenido por GitHub. Brackets está disponible para Mac, Windows y Linux.




\section{Equipo de proyecto}
En este apartado se describirá al equipo del proyecto desde dos perspectivas diferentes. En la sección de Interacción con el cliente se describirá como deberá ser el contacto con el cliente poniendo en práctica las buenas prácticas de la metodología Scrum. En el caso de los perfiles profesionales se definirán los perfiles que deberá haber para desarrollar el proyecto y cómo será el organigrama del proyecto.

\subsection{Interacción con el cliente}
El esquema organizativo

\begin{itemize}
	\item \textbf{Product Owner:} Este rol será representado por una persona de Alboan. Sus principales funciones son las siguientes:
		\subitem Ser el representante de todas las personas interesadas en los resultados del proyecto
		\subitem Definir los objetivos del producto o proyecto.
		\subitem Colaborar con el equipo para planificar, revisar y dar detalle a los objetivos de cada iteración.
	\item \textbf{Scrum master:} No se debe confundir con el jefe de proyecto. Sus principales tareas son:
		\subitem Facilitar las reuniones de Scrum
		\subitem Proteger y aislar al equipo de interrupciones externas
		\subitem Quitar los impedimentos que el equipo tiene en su camino
	\item \textbf{Equipo de desarrollo:} Compuesto por los profesionales que desarrollarán el proyecto. Su principal función es desarrollar el proyecto y estas son otras de sus funciones:
		\subitem Equipo auto organizado.
		\subitem Equipo multidisciplinar, capacidad de desarrollar diferentes tareas.
		\subitem Equipo estable durante el proyecto y establecidos en la misma localización física.

\end{itemize}

En la figura 5.1 se puede ver cómo será la interlocución entre los miembros del equipo. El Scrum master, representado por Rubén Sánchez, será el encargado de interaccionar con el Product Owner, que estará representado por una persona de Alboan. Por último, el equipo de desarrollo incluirá a las personas que desarrollen el proyecto, las cuales podremos ver más adelante.

\figura{0.6}{imgs/equiposcrum.png}{Gráfico de la interacción en el proyecto}{equiposcrum}{}

\subsection{Perfiles profesionales}

En este apartado se describirá al equipo de proyecto que será el encargado de desarrollar el producto y los diferentes perfiles necesarios para ello.

\begin{itemize}
	\item \textbf{Jefe de proyecto:} Es el encargado de controlar que el proyecto se desarrolle correctamente y de garantizar los tiempos de desarrollo y la calidad del mismo. También se encargará de los aspectos de gestión del proyecto.
	\item \textbf{Diseñador:} Es el encargado de diseñar la interfaz de la página web y del widget.
	\item \textbf{Analista de datos:} Es el encargado de analizar los datos que se extraen del proyecto y de crear informes interactivos y conclusiones de ellos.
	\item \textbf{Arquitecto web:} Es el encargado de diseñar y desarrollar la arquitectura principal del proyecto. También se encarga de configurar las herramientas que vayan a utilizar los demás para asegurar su correcto funcionamiento.
	\item \textbf{Programador:} Es el encargado de desarrollar los métodos que el arquitecto haya establecido.
\end{itemize}

En la figura 5.2 se ve la organización en el equipo de proyecto. Alboan está como cliente externo a Deusto Tech. Dentro de Deusto Tech tenemos a Pablo García, el cual será el supervisor del proyecto desde la empresa, y al equipo de la Colmena. Dentro del equipo de la Colmena contamos con dos personas, el jefe del proyecto, que hará las veces de Diseñador, Analista de datos y Arquitecto Web y, por otra parte, el programador.

\figura{0.8}{imgs/Equipo.png}{Gráfico del equipo de proyecto}{equipo}{}

\subsection{Procedimientos de seguimiento y control}

El seguimiento del proyecto se hará desde el tablón de Trello. En el tablón se registran los cambios que se generan y se generan notificaciones para las personas involucradas en la tarea que ha sufrido cambios. Finalmente se puede sacar un reporte de los cambios en las tareas.\\

En cuanto a reuniones de seguimiento del proyecto, tendremos una reunión cada viernes en la que el equipo de proyecto hablará con el supervisor de Deusto Tech. En esa reunión se podrán ver los avances y problemas que hay cada semana y se pensarán diferentes soluciones para ellos. Estas reuniones también servirían por si se necesitará algo de Deusto Tech.


\chapter{Planificación}


En este capítulo hablaremos sobre la planificación y presupuestos del proyecto.

\section{Presupuesto}
En este apartado se detalla el coste en días de cada tarea. Cada día está compuesto por 8h laborales. Cada tarea será realizada por un perfil profesional, cobrando cada uno su precio en horas.

\figuraSinMarco{1}{imgs/tareas.png}{Lista de tareas con su coste en horas}{tareas}{}

\begin{table}[]
	\centering
	\begin{tabular}{llll}
		\textbf{Perfil profesional} & \textbf{Horas} & \textbf{Precio (por horas)} & \textbf{Importe} \\
		Diseñador                   & 120            & 20                          & 2400             \\
		Arquitecto web              & 136            & 20                          & 2720             \\
		Programador                 & 240            & 20                          & 4800             \\
		Analista de datos           & 120            & 25                          & 3000             \\
		\hline Precio total                &                &                             & 12920
	\end{tabular}
	\caption{Presupuesto del proyecto desglosado}
	\label{precios}
\end{table}


\section{Planificación del proyecto}

En la figura 6.2 se puede ver el diagrama Gantt del proyecto. Este está divido en los 4 sprints de los que antes hemos hablado. Las barras negras corresponden a los sprints, cada uno de los sprints esta compuesto por unas barras que indican la duración de las tareas dentro de ellos. Las flechas indican la precedencia de las tareas

\figuraSinMarco{0.7}{imgs/gantt.png}{Diagrama Gantt}{gantt}{}

\section{Justificación financiera}

Este proyecto no genera beneficios ya que se trata de una plataforma para obtener donaciones para otros proyectos, por lo que el dinero recaudado por la plataforma está directamente destinado a los proyectos. Por otra parte, la plataforma ofrecerá numerosos beneficios a la ONG ya que hará más visibles sus proyectos y ofrecerá una manera más fácil de hacer donaciones a las personas que estén interesadas. También unificará la manera de recibir estas donaciones y las tendrá todas almacenadas en un sistema completo por lo que hacer estudios sobre estas donaciones será muy sencillo.


\chapter{Especificación de requisitos}

\section{Visión general}

En este capítulo se especifican los requisitos que el proyecto debe satisfacer y que definen el funcionamiento de todo el software que compone este proyecto. Para una mejor compresión de los mismos, se dividen en los siguientes bloques:

\begin{itemize}
	\item \textit{Especificación de requisitos del servidor Node.js:} en esta sección se recogen los requisitos que debe de satisfacer el servidor, este es el encargado de soportar todo el sistema.
	\item \textit{Especificación de requisitos de la página web:} en esta sección se recogen los requisitos que debe satisfacer la página web del proyecto.
	\item \textit{Especificación de requisitos del widget:} en esta sección se recogen los requisitos que debe satisfacer el widget del proyecto, el elemento que estará incrustado en las tiendas online.
	\item \textit{Especificación de requisitos de la base de datos:} en esta sección se recogen los requisitos que debe satisfacer la base de datos, esta albergará los datos de las donaciones y donantes.
	\item \textit{Especificación de requisitos del sistema de visualización:} en esta sección se recogen los requisitos que debe satisfacer el sistema de visualización, el encargado de crear gráficas interactivas para la visualización y estudio de los datos.
\end{itemize}

\section{Especificación de requisitos del servidor Node.js}
Los requisitos funcionales del servidor Node.js son:

\begin{itemize}
	\item \textbf{RF.0.1} El servidor debe ser capaz de albergar la página web.
	\item \textbf{RF.0.2} El servidor debe ser capaz de albergar las rutas de la página web y ofrecer el enrutamiento a cada una de estas.
	\item \textbf{RF.0.3} El servidor debe ser capaz de conectarse con la base de datos.
	\item \textbf{RF.0.4} El servidor debe ser capaz de enviar mails.
	\item \textbf{RF.0.5} El servidor debe ser capaz de alterar un PDF.
	\item \textbf{RF.0.6} El servidor debe ser capaz de crear un script y almacenarlo.
	\item \textbf{RF.0.7} El servidor debe ser capaz de ofrecer un script a páginas de terceros.
	\item \textbf{RF.0.8} El servidor debe ser capaz de ofrecer datos a una tercera aplicación.
\end{itemize}

Los requisitos no funcionales son:

\begin{itemize}
	\item \textbf{RNF.0.1} Mantenibilidad: el sistema tiene que tener un mantenimiento sencillo ya que tiene conexiones con páginas de terceros lo que obliga a que el mantenimiento sea sencillo y rápido.
	\item \textbf{RNF.0.2} Escalabilidad: el servidor tiene que ser escalable ya que la creación de nuevos widgets o la oferta de datos a terceras aplicaciones puede ser grande.
\end{itemize}



\section{Especificación de requisitos de la página web}

Los requisitos funcionales de la página web son:

\begin{itemize}
	\item \textbf{RF.1.1} La página web debe ofrecer un asistente para la creación de nuevos. widgets
	\item \textbf{RF.1.2} La página web debe ofrecer un formulario para la obtención de un certificado de donación.
	\item \textbf{RF.1.3} La página web debe ofrecer un sistema para comunicarse con el soporte del proyecto.
\end{itemize}

Los requisitos no funcionales de la página web son:

\begin{itemize}
	\item \textbf{RNF.1.1} La página web debe informar sobre el proyecto.
	\item \textbf{RNF.1.2} La página web debe informar sobre los proyectos de la ONG.
	\item \textbf{RNF.1.3} Accesibilidad: La página web tiene que ser responsiva para ser adaptable en diferentes dispositivos.
	\item \textbf{RNF.1.4} Interfaz: la página web debe tener un diseño simple para facilitar la navegación.
	\item \textbf{RNF.1.5} Escalabilidad: la página tiene que ser escalable ya que se tienen que poder añadir nuevos proyectos a ella.
\end{itemize}

\section{Especificación de requisitos del widget}

Los requisitos funcionales del widget son:

\begin{itemize}
	\item \textbf{RF.2.1} El widget debe permitir la donación fija o variable de una cantidad de dinero.
	\item \textbf{RF.2.2} El widget debe ser de fácil implantación por parte de la tienda online.
\end{itemize}

Los requisitos no funcionales del widget son:

\begin{itemize}
	\item \textbf{RNF.2.1} El widget debe informar sobre el proyecto al que va destinado.
	\item \textbf{RNF.2.2} Accesibilidad: el widget debe ser responsivo para que pueda añadirse en cualquier tienda.
	\item \textbf{RNF.2.3} Interfaz: el widget debe ser 100\% personalizable.
	\item \textbf{RNF.2.4} Disponibilidad: el widget debe estar disponible en todo momento para su uso por parte de los comercios online.
\end{itemize}

\section{Especificación de requisitos de la base de datos}

Los requisitos funcionales de la base de datos son:

\begin{itemize}
	\item \textbf{RF.3.1} La base de datos debe almacenar los datos de las donaciones.
	\item \textbf{RF.3.2} La base de datos debe almacenar los dados de los donantes.
	\item \textbf{RF.3.3} La base de datos debe proporcionar los datos que se le pidan.
\end{itemize}

Los requisitos no funcionales de la base de datos son:

\begin{itemize}
	\item \textbf{RNF.3.1} Rendimiento: la base de datos debe almacenar y proporcionar los datos en un tiempo razonable.
	\item \textbf{RNF.3.2} Seguridad: la base de datos debe ser segura para no poner en peligro los datos de los donantes.
	\item \textbf{RNF.3.3} Disponibilidad: La base de datos debe estar disponible para almacenar las donaciones e información de donantes.
\end{itemize}

\section{Especificación de requisitos del sistema de visualización}

Los requisitos funcionales del sistema de visualización son:

\begin{itemize}
	\item \textbf{RF.4.1} El sistema de visualización debe ofrecer gráficos interactivos.
	\item \textbf{RF.4.2} El sistema de visualización debe conectarse con el servidor para adquirir los datos.
\end{itemize}

Los requisitos no funcionales del sistema de visualización son:

\begin{itemize}
		\item \textbf{RNF.4.1} Escalabilidad: el sistema de visualización de datos debe ser escalable ya que los datos pueden crecer y la manera de mostrarlos cambiar.
		\item \textbf{RNF.4.2} Interfaz: el sistema de visualización de datos debe tener una interfaz intuitiva que permita una buena navegación por los gráficos.
\end{itemize}


\chapter{Tecnologías utilizadas}


% Apéndices
\backmatter
\appendix
%\chapter{Acrónimos}

%\chapter{Licencia}
%\input{0202_apendice}

\bibliografia{otrasreferencias}
%\bibliografiaOtras{otrasreferencias.bib} %Opcional


\end{document}
