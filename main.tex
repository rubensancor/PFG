\documentclass{memoriaPFC}
\fontfamily{cmss}\selectfont



%
% DATOS DEL DOCUMENTO
%

\title{Colmena: diseño e implementación de un widget web para micro donaciones, con página web de soporte, almacenamiento de datos y su posterior visualización}

%\autores{Nombre1 Apellidos1}{Nombre2 Apellidos2} Hasta 5 autores
\autores{Rubén Sánchez Corcuera}{}{}{}{}

%El director del proyecto
\director{Pablo García Bringas}
%\directora{Ursula K. Le Guin}

% iaei ii ioi itel (ii es un pdf que contiene los logos para el grado de informatica, el unico actualizado)
\titulacion{ii}

% mayo || septiembre (en minúsculas!!)
\date{mayo de 2017}

\resumen{%
 El proyecto consta de varios módulos. La base de todo es un servidor implementado en node.js. En él se aloja la página web, los widgets de las diferentes empresas que han decidido implantarlo en sus páginas web y una API que administra el enrutamiento de la web y permite varias funciones sobre la base de datos.\\
 
 La página web sirve de apoyo al widget, es donde se muestra la información general del widget, los diferentes proyectos a los que apoyar, un wizard con el que poder crear tu propio widget y la plataforma para poder conseguir el certificado de donación.\\
 
 La base de datos noSQL aloja los datos de las diferentes donaciones para posteriormente poder crear el certificado de donación y eventualmente generar estadísticas sobre las cantidades, fechas y lugares en los que más donaciones se están obteniendo.\\
 
 Por último, el widget. Está pensado para que las empresas que tengan portales de compra online lo incrusten durante algunas de sus fases compra del cliente con el fin de que este haga una pequeña donación a la causa que las empresas han “apadrinado”. El widget es personalizable por las empresas que lo quieran implantar en su web, así conseguir una mayor integración con ellas. Entre sus opciones de personalización destaca la posibilidad de elegir entre una cantidad fija o variable de donación.\\
 
 
 Finalmente tiene implementado un sistema con graficos para extraer estadisticas de las donaciones.
  
 
}
\descriptores{solidaridad, widget, NodeJS, MongoDB, micro donaciones}


%
% COMIENZO DEL DOCUMENTO
%
\begin{document}

% Portada, resumen, indices
\frontmatter
\hacerportada
\hacerresumen
\tableofcontents
\listoffigures % Opcional
\listoftables % Opcional
\lstlistoflistings % Opcional

% Contenidos
\mainmatter
\chapter{Introducción}

\section{Presentación del documento}

El presente documento recoge la definición de objetivos del proyecto que realizará el alumno Rubén Sánchez para la ONG ALBOAN. En el proyecto consta de todo el sistema necesario para crear un sistema de micro donaciones mediante un widget 5 en cualquier tienda online. El sistema consta de: una página web, una base de datos, el widget y un servidor que da soporte a todo. Además, el sistema cuenta con una funcionalidad añadida por la que se puede crear gráficas de visualización.\\

El proyecto se desarrolla para la ONG Alboan la cual al ver el cambio que se da en la sociedad en materia de solidaridad y que las grandes donaciones han sufrido una gran caída decide que la manera de recoger las donaciones tiene que cambiar. Alboan, al explorar las oportunidades y ver que la única opción disponible cobra a las ONGs un porcentaje de las donaciones decide crear un sistema de micro donaciones gratuito que cualquier empresa pueda añadir a su tienda online y así fomentar las donaciones a pequeña escala, consiguiendo así que el 100\% del dinero vaya al destino final.\\

Los principales capítulos del documento son los siguientes:

\begin{itemize}
	\item \textbf{Introducción} \smallbreak
	 En este apartado se incluye una visión global del documento y del proyecto que se va a llevar a cabo.
	\item \textbf{Demandante del proyecto} \smallbreak
	 En este apartado se hará una descripción sobre la empresa demandante.
	 \item \textbf{Situación de partida} \smallbreak
	 En este apartado se trata la motivación por la que se va a desarrollar el proyecto, con esto también se analizan las posibles soluciones dadas para el problema y por últimos los objetivos que se quieren conseguir.
	\item \textbf{Definición del proyecto}\smallbreak
	Este es el apartado más importante del documento, en él se describe el producto final que se va a generar y la metodología utilizada para desarrollarlo.
	\item \textbf{Condiciones de ejecución}\smallbreak
	En este apartado se describen los materiales utilizados para desarrollar el proyecto, el hardware necesario para su puesta en marcha y el equipo que lo desarrollará.
	\item \textbf{Presupuestos y tiempos} \smallbreak
	En este apartado se describen los gastos que puede tener el proyecto en materia de tareas y equipo utilizado para desarrollarlo.
\end{itemize}


\chapter{Antecedentes y Justificación}

En este capitulo se hablará sobre el cliente que pide el proyecto y el estado del arte del mismo.

\section{Demandante del proyecto}
En esta sección se hablará el demandante del proyecto, en este caso es la ONG Alboan.

\subsection{Alboan}

La actividad de Alboan\cite{alboan} comenzó en 1994, asumiendo las iniciativas de voluntariado
internacional que ya estaban en marcha, pero fue en 1996 cuando se configuró
jurídicamente bajo la figura de Fundación. Es la ONG promovida por la Compañía de Jesús en la Provincia de Loyola (País Vasco y Navarra). El nombre, en euskera, Alboan, quiere reflejar el arraigo a la cultura de la tierra vasca y el espíritu de la entidad: estar al lado de las personas más excluidas, junto a organizaciones y centros educativos.\\

Su logo (Figura \ref{logo}) visibiliza  el rol de bisagra y puente que quería jugar la institución para poner en relación dos mundos que en realidad son uno solo. Su misión: ser plataforma de encuentro de personas y organizaciones de aquí y allá que quisieran comprometerse en la construcción de un mundo mejor.\\

Hoy Alboan cuenta con 7000 personas entre contratados, voluntarios y entidades que le permiten apoyar y acompañar a más de 200 proyectos llevados a cabo por 105 organizaciones
aliadas que impactan directamente en las vidas de más de 600.000 personas en todo
el mundo.

\figuraSinMarco{0.5}{imgs/alboan.jpg}{Logo de Alboan}{logo}{}

Trabaja por la construcción de una ciudadanía global que denuncie las injusticias que provocan desigualdad en el mundo, construya una cultura que promueva el bien común y transforme las estructuras generadoras de pobreza a nivel local y global. Para lograrlo, se une en red con personas y grupos de todo el mundo.\\

La colaboración de Alboan se centra en 4 temáticas principales en las que enmarca todos sus proyectos y labores de ayuda: Educación de calidad, Desarrollo económico-productivo sostenible y equitativo, Acción humanitaria en crisis recurrentes y Democracia a favor de las personas excluidas.\\

Todas las actividades que realiza la ONG incluyen 3 ejes transversales que otorgan a los proyectos el carisma que Alboan quiere dar:
\begin{itemize}
	\item La espiritualidad como dimensión en el horizonte de desarrollo humano.
	\item El reconocimiento de las desigualdades entre mujeres y hombres y el compromiso con la equidad de género.
	\item La participación ciudadana para la incidencia social y política.
\end{itemize}

\section{Estado del arte}
En este apartado se hace un análisis de las diferentes opciones que hay para cubrir las necesidades citadas anteriormente.
\subsection{Crowdfunding}
“Cooperación colectiva, llevada a cabo por personas que realizan una red para conseguir dinero u otros recursos, se suele utilizar Internet para financiar esfuerzos e iniciativas de otras personas u organizaciones”\cite{crowdw}\\

En el crowdfundig hay diferentes tipos de ayudas y de sistemas creados para financiar los proyectos que la gente sube a las plataformas: basado en donaciones (migranodearena, teaming...), basados en recompensas(Goteo, Verkami...) y para préstamos(ECrowd!) entre otros.

\subsubsection{Basado en donaciones}
El usuario realiza una donación para ayudar a un proyecto que le gusta y no recibe ningún beneficio económico en retorno (solo la satisfacción de hacer algo bueno y de apoyar un proyecto que le emociona). Son proyectos solidarios con un impacto social grande.
\begin{itemize}
	\item Migranodearena.org
	\item Teaming
\end{itemize}

\subsubsection{Basado en recompensas}
Se trata de aportar dinero a un proyecto y se recibe a cambio una recompensa. Los proyectos pueden consistir en crear y diseñar un nuevo producto o financiar un proyecto cultural (película, festival, musical). Es el modelo de crowdfunding más utilizado y conocido. En cada proyecto se puede elegir entre un rango de recompensas en función de la cantidad de dinero que se aporte.
	
\begin{itemize}
	\item Goteo
	\item Verkami
\end{itemize}
		
\subsubsection{Para prestamos}
El usuario realiza una inversión en un proyecto o en una empresa de la plataforma y recibe un retorno económico en forma de intereses. La empresa contrata un préstamo con la plataforma de crowdlending, que lo gestiona, y el usuario recupera el dinero invertido junto a los intereses a lo largo del tiempo según las condiciones pactadas en el préstamo.
		
\begin{itemize}
	\item ECrowd!
\end{itemize}
			
\subsection{Micro donaciones}
Las micro donaciones, como su nombre indica, son donaciones muy pequeñas realizadas con el fin de aportar una pequeña cantidad de dinero al proyecto que nos interesa. En este ámbito nos encontramos con Worldcoo\cite{worldcoo} una herramienta online con bastante recorrido y reconocimiento.
			
\begin{itemize}
	\item \textbf{Worldcoo:} \smallbreak
	Worldcoo es una empresa nacida en 2012 que pretende crear un nuevo canal de financiación mediante un widget incrustado en tiendas online. Esto permite que las miles de personas que acceden y compran en las tiendas online puedan donar una pequeña cantidad al proyecto que esa página tiene apadrinado.\\
	
	Cuentan con un extenso equipo de personas y de embajadores en varios paises del mundo, lo que les da acceso a muchos comercios online. Las empresas contactan con Worldcoo y acuerdan entre los dos la instalación del widget en su tienda online. Este widget esta relacionado con uno de los proyectos de las ONGs a las que Worldcoo ayuda.
\end{itemize}
				
Esto es exactamente lo que Alboan quiere, pero tiene algunas carencias. La empresa Worldcoo se queda un 8\% de las donaciones realizadas a cada proyecto por lo que esto empieza a disgustar a Alboan, esta empresa tampoco permite ver las personas que están donando a los proyectos por lo que no permite mantener el contacto con esas personas para hacerles conscientes de que su ayuda tiene un resultado. \\

Después de hacer un análisis de todas las opciones y ver los resultados Alboan decide crear su propia plataforma de micro donaciones mediante widgets en tiendas online.

\chapter{Objetivos del proyecto}

\section{Definición del proyecto}

\subsection{Objetivos}
El principal objetivo de este proyecto es crear un widget solidario que se pueda implantar en cualquier tienda online y crear la página web en la que se va a apoyar y se dará a conocer este widget. La página web también será el portal en el que la gente que ha realizado alguna donación pueda recoger su certificado de donación con el fin de presentarlo en la declaración de la renta.\\

Otro objetivo sería el de crear el servidor con la base de datos para alojar los datos de las donaciones realizadas mediante el widget. Este servidor seria accesible para las personas de Alboan que quisieran consultar los datos de las donaciones o ver las gráficas relacionadas con esto.

\subsection{Alcance del proyecto}
Teniendo en cuenta los objetivos citado anteriormente la Colmena deberá cumplir con las siguientes funcionalidades:

\begin{itemize}
	\item Una página web responsiva que informe sobre todo del proyecto Colmena pero que también tenga información sobre la ONG y que tenga algún enlace para redirigir a esta.
	\item Un widget que permita hacer micro donaciones a los proyectos de Alboan y que informe sobre el proyecto al que va destinado el dinero.
	\item Una base de datos en la que almacenar las donaciones y los datos de los donantes.
	\item Una herramienta con la que poder dispensar certificados de donación a las personas que han aportado con los proyectos.
	\item Una herramienta para poder personalizar el widget y generarlo automáticamente.
	\item Un servidor en el que centralizar todas las funcionalidades y rutas de la solución. Además el servidor será capaz de ofrecer una API para aplicaciones futuras que quieran añadirse a la solución.
	\item Un sistema de visualización interactivo en el que poder analizar los datos obtenidos desde la aplicación y sacar conclusiones de ellos.
\end{itemize}

El proyecto no se encargará de hacer llegar el dinero de las donaciones, desde los donantes hasta la ONG, sino que hará de puente contabilizando y manteniendo un registro de estas para que finalmente el comercio online pueda hacer llegar este dinero a la entidad.

\subsection{Producto final}
El producto final se compone de varios elementos listados a continuación:\\

El elemento visual del proyecto lo formarían \textbf{la página web y el widget}. Estos dos elementos permitirán a los usuarios del sistema informarse de los proyectos que la ONG realiza, contactar con el soporte del proyecto o hacer uso de las funcionalidades que ofrecen, como por ejemplo recibir un certificado de donación o hacer una donación a uno de los proyectos. El widget puede estar alojado en cualquier comercio online. Estos dos elementos están estrictamente ligados a el servidor del proyecto que es el que les permite implementar todas sus funcionalidades.\\

El elemento central del proyecto esta formado por \textbf{el servidor y la base de datos}. Estos dos elementos permiten desarrollar toda la funcionalidad del sistema ya que ofrecen un sistema de enrutado para la parte visual y toda la funcionalidad que deban implementar sus elementos. Por otra parte, este elemento ofrecerá una API en la que se ofrecen diferentes funcionalidades como consultas a la base de datos, siempre con cierta privacidad hacia los datos sensibles.\\

El elemento final del proyecto consta del \textbf{sistema de visualización} de los datos. Gracias a este sistema se podrán visualizar los datos de manera interactiva y permitir a los empleados de Alboan hacer reflexiones sobre sus proyectos o los diferentes sectores de la población que les apoya en diferente medida. Por otra parte, también permitirá a los donantes ver como es la sociedad que les rodea y hacer un análisis de ella.

\section{Descripción de realización}

\subsection{Método de desarrollo}

\subsection{Productos intermedios}
Gracias a la metodología que se va a utilizar es muy fácil generar productos intermedios. Durante el proceso de desarrollo del proyecto se crearán varios prototipos/mockups del widget y de la página web. Al final de cada sprint se hará una revisión del trabajo que se ha realizado, esto será posible gracias al tablón Trello en el que se podrán ver las tareas realizadas. Para definir los productos intermedios que se van a generar se procede a listarlos a continuación:

\begin{itemize}
	\item Documento sobre las tecnologías a usar
	\item Documento de seguimiento de las tareas realizadas
	\item Varios prototipos de la página web
	\item Varios prototipos del widget
	\item Manual de usuario para los técnicos que mantendrán el proyecto
	\item Documento de conclusiones sobre las donaciones realizadas
\end{itemize}

Gracias a estos productos intermedios se podrá mantener un seguimiento del proyecto y tener una mejor estimación de las tareas y los tiempos necesarios para estas.


\subsection{Tareas principales}

\section{Organización y equipo}

\subsection{Esquema organizativo}

\subsection{Plan de recursos humanos}

\section{Condiciones de ejecución}

\subsection{Entorno de trabajo}

\subsection{Control de cambios}

\subsection{Recepción de productos}

\section{Planificación}

\subsection{Diagrama de precedencias}

\subsection{Plan de trabajo}

\subsection{Diagrama de Gantt}

\subsection{Estimación de cargas de trabajo}

\section{Presupuesto}

\chapter{Especificación de requisitos}

\section{Visión general}

En este capitulo se especifican los requisitos que el proyecto debe satisfacer y que definen el funcionamiento de todo el software que compone este proyecto. Para una mejor compresión de los mismos, se dividen en los siguientes bloques:

\begin{itemize}
	\item \textit{Especificación de requisitos del servidor Node.js:} en esta sección se recogen los requisitos que debe de satisfacer el servidor, este es el encargado de soportar todo el sistema.
	\item \textit{Especificación de requisitos de la página web:} en esta sección se recogen los requisitos que debe satisfacer la página web del proyecto.
	\item \textit{Especificación de requisitos del widget:} en esta sección se recogen los requisitos que debe satisfacer el widget del proyecto, el elemento que estará incrustado en las tiendas online.
	\item \textit{Especificación de requisitos de la base de datos:} en esta sección se recogen los requisitos que debe satisfacer la base de datos, esta albergará los datos de las donaciones y donantes.
	\item \textit{Especificación de requisitos del sistema de visualización:} en esta sección se recogen los requisitos que debe satisfacer el sistema de visualización, el encargado de crear gráficas interactivas para la visualización y estudio de los datos.
\end{itemize}

\section{Especificación de requisitos del servidor Node.js}
Los requisitos funcionales del servidor Node.js son:

\begin{itemize}
	\item \textbf{RF.0.1} El servidor debe ser capaz de albergar la página web.
	\item \textbf{RF.0.2} El servidor debe ser capaz de albergar las rutas de la página web y ofrecer el enrutamiento a cada una de estas.
	\item \textbf{RF.0.3} El servidor debe ser capaz de conectarse con la base de datos.
	\item \textbf{RF.0.4} El servidor debe ser capaz de enviar mails.
	\item \textbf{RF.0.5} El servidor debe ser capaz de alterar un PDF.
	\item \textbf{RF.0.6} El servidor debe ser capaz de crear un script y almacenarlo.
	\item \textbf{RF.0.7} El servidor debe ser capaz de ofrecer un script a páginas de terceros.
	\item \textbf{RF.0.8} El servidor debe ser capaz de ofrecer datos a una tercera aplicación.
\end{itemize}

Los requisitos no funcionales son:

\begin{itemize}
	\item \textbf{RNF.0.1} Mantenibilidad: el sistema tiene que tener un mantenimiento sencillo ya que tiene conexiones con paginas de terceros lo que obliga a que el mantenimiento sea sencillo y rápido.
	\item \textbf{RNF.0.2} Escalabilidad: el servidor tiene que ser escalable ya que la creación de nuevos widgets o la oferta de datos a terceras aplicaciones puede ser grande.
\end{itemize}



\section{Especificación de requisitos de la página web}

Los requisitos funcionales de la página web son:

\begin{itemize}
	\item \textbf{RF.1.1} La página web debe ofrecer un wizard para la creación de nuevos. widgets
	\item \textbf{RF.1.2} La página web debe ofrecer un formulario para la obtención de un certificado de donación.
	\item \textbf{RF.1.3} La página web debe ofrecer un sistema para comunicarse con el soporte del proyecto.
\end{itemize}

Los requisitos no funcionales de la página web son:

\begin{itemize}
	\item \textbf{RNF.1.1} La página web debe informar sobre el proyecto.
	\item \textbf{RNF.1.2} La página web debe informar sobre los proyectos de la ONG.
	\item \textbf{RNF.1.3} Accesibilidad: La página web tiene que ser responsiva para ser adaptable en diferentes dispositivos.
	\item \textbf{RNF.1.4} Interfaz: la página web debe tener un diseño simple para facilitar la navegación.
	\item \textbf{RNF.1.5} Escalabilidad: la página tiene que ser escalable ya que se tienen que poder añadir nuevos proyectos a ella.
\end{itemize}

\section{Especificación de requisitos del widget}

Los requisitos funcionales del widget son:

\begin{itemize}
	\item \textbf{RF.2.1} El widget debe permitir la donación fija o variable de una cantidad de dinero.
	\item \textbf{RF.2.2} El widget debe ser de fácil implantación por parte de la tienda online.
\end{itemize}

Los requisitos no funcionales del widget son:

\begin{itemize}
	\item \textbf{RNF.2.1} El widget debe informar sobre el proyecto al que va destinado.
	\item \textbf{RNF.2.2} Accesibilidad: el widget debe ser responsivo para que pueda añadirse en cualquier tienda.
	\item \textbf{RNF.2.3} Interfaz: el widget debe ser 100\% personalizable.
	\item \textbf{RNF.2.4} Disponibilidad: el widget debe estar disponible en todo momento para su uso por parte de los comercios online.
\end{itemize}

\section{Especificación de requisitos de la base de datos}

Los requisitos funcionales de la base de datos son:

\begin{itemize}
	\item \textbf{RF.3.1} La base de datos debe almacenar los datos de las donaciones.
	\item \textbf{RF.3.2} La base de datos debe almacenar los dados de los donantes.
	\item \textbf{RF.3.3} La base de datos debe proporcionar los datos que se le pidan.
\end{itemize}

Los requisitos no funcionales de la base de datos son:

\begin{itemize}
	\item \textbf{RNF.3.1} Rendimiento: la base de datos debe almacenar y proporcionar los datos en un tiempo razonable.
	\item \textbf{RNF.3.2} Seguridad: la base de datos debe ser segura para no poner en peligro los datos de los donantes.
	\item \textbf{RNF.3.3} Disponibilidad: La base de datos debe estar disponible para almacenar las donaciones e información de donantes.
\end{itemize}

\section{Especificación de requisitos del sistema de visualización}

Los requisitos funcionales del sistema de visualización son:

\begin{itemize}
	\item \textbf{RF.4.1} El sistema de visualización debe ofrecer gráficos interactivos.	
	\item \textbf{RF.4.2} El sistema de visualización debe conectarse con el servidor para adquirir los datos.
\end{itemize}

Los requisitos no funcionales del sistema de visualización son:

\begin{itemize}
		\item \textbf{RNF.4.1} Escalabilidad: el sistema de visualización de datos debe ser escalable ya que los datos pueden crecer y la manera de mostrarlos cambiar.
		\item \textbf{RNF.4.2} Interfaz: el sistema de visualización de datos debe tener una interfaz intuitiva que permita una buena navegación por los gráficos.
\end{itemize}

\chapter{Tecnologías utilizadas}

En esta sección se presentan las diferentes tecnologías que se han usado para el desarrollo del proyecto. Las principales tecnologías utilizadas son las siguientes:

\section{Node.js}
Esta es la autodefinición que se hace Node.js en su propia página web:

\begin{quote}
	Node.js\cite{wsdl13} es un entorno de ejecución para JavaScript construido con el motor de JavaScript V8 de Chrome. Node.js usa un modelo de operaciones E/S sin bloqueo y orientado a eventos, que lo hace liviano y eficiente. El ecosistema de paquetes de Node.js, npm, es el ecosistema mas grande de librerías de código abierto en el mundo.\\
\end{quote}


Node.js es una solución con un solo hilo de ejecución que permite que las peticiones a esta no bloqueen peticiones futuras ni exige grandes pools de hilos para tener conexiones concurrentes. Esto permite que los comercios onlines conecten con el servidor y le hagan una petición para recibir el widget a lo que el servidor responderá con el envió y el cierre de la conexión, evitando así el cuello de botella que se puede generar con conexiones masivas.\\

Node.js cuenta con un gestor de paquetes llamado npm del que se pueden descargar diferentes modulos para ampliar la funcionalidad de esta tecnología. Gracias a este gestor de paquetes node.js se convierte en una solución integral para la parte servidora habilitándole con todo lo necesario para cumplir las funciones del backend de una solución web.

Entre las ventajas que ofrece Node.js se encuentran las siguientes:

\begin{itemize}
	\item \textbf{Gran documentación:} tiene una gran documentación y 8 años de experiencia por lo que la mayoría de los casos y posibilidades están testadas haciendo su desarrollo mas sencillo.
	\item \textbf{Gran comunidad:} gracias al gestor de paquetes publico y a los años de experiencia Nodejs cuenta con una gran comunidad con la que poder consultar las dudas y usos de los diferentes paquetes.
	\item \textbf{npm:} su gestor de paquetes publico  permite reutilizar código y no perder tiempo implementando código que ya ha sido desarrollado y probado anteriormente.
	\item \textbf{Multiplataforma y open-source:} esta desarrollado para ser utilizado en cualquier sistema operativo y cuenta con una licencia MIT, lo cual lo hace gratuito y permite arreglar e incluso mejorar el propio código de la herramienta por los usuarios de esta.
\end{itemize}

\subsection{Funcionamiento}
En Node.js es muy sencillo crear aplicaciones nuevas que actuen en la parte servidora de una aplicación. Gracias a npm, Node.js es muy polivalente, pero a continuación se mostrará un ejemplo de un servidor HTTP:

\begin{codigo}
http.createServer(function (request, response) \{
	response.writeHead(200, {'Content-Type': 'text/plain'});
	response.end('Hello World');
\}).listen(8081);
console.log('Server running at http://127.0.0.1:8081/');
\end{codigo}

En este ejemplo se crea un servidor HTTP en el puerto 8081 de la maquina local. Una vez se entra al puerto 8081 de la maquina local, el servidor creara una respuesta en texto plano y la enviará al navegador, que mostrará por pantalla el mensaje.\\

Node.js funciona de una manera muy diferente dependiendo de los paquetes utilizados para el desarrollo de la aplicación por lo que a continuación explicare los paquetes utilizados para el desarrollo de este proyecto y como es el funcionamiento de cada uno de estos:

\subsection{Express}
Express es un framework web minimalista y flexible para el desarrollo de soluciones web en node.js. Express además aplica una fina capa con las características fundamentales de las aplicaciones web sobre la base de node.js permitiendo así mantener todas las funcionalidades que ofrece Node.js. Finalmente, Express ofrece una amplia y robusta API para hacer uso de todas las funcionalidades y características que ofrece.

\subsubsection{Funcionamiento}
Express no tiene una estructura de proyecto definida por los desarrolladores o comunidad, en cambio, tanto en la documentación como en numerosos sitios ofrecen un sistema de carpetas para tener el proyecto ordenado y las rutas definidas.

\begin{codigo}
	project/
	|--	node\_modules/
	|--	public/
			 |--	images/
			 |--	css/
			 |--	javascript/
	|--	routes/
	|--	views/
	|--	app.js
	|--	package.json
\end{codigo}

\begin{itemize}
	\item \textbf{Node\_modules:} en esta carpeta irán incluidas todas las dependencias del proyecto una vez descargadas automáticamente por npm.
	\item \textbf{Public:} en esta carpeta estarán los elementos de uso publico por parte del proyecto, imágenes, scripts de JavaScript y hojas de estilo.
	\item \textbf{Routes:} aquí se encuentran las rutas a las diferentes entidades en archivos diferentes.
	\item \textbf{Views:} en esta carpeta estarán incluidas las vistas del proyecto, es decir, las diferentes pantallas de la página web.
	\item \textbf{App.js:} es el archivo principal del proyecto.
	\item \textbf{Package.json:} es un archivo de configuración general del proyecto, en el irán las dependencias y los metadatos del proyecto.
\end{itemize}

\subsubsection{Razón de uso}
Se ha elegido este framework ante otros disponibles para Node.js por varias razones:

\begin{itemize}
	\item El primer commit de esta herramienta fue 2 meses después de la creación de Node.js y la primera versión 1 año después por lo que tiene un gran recorrido y madurez.
	\item Su sencillez para manejar las rutas y las vistas en una solución web.
	\item Gran comunidad de desarrolladores en la que apoyarse.
\end{itemize}

\subsection{Express-ejs-layouts}

\subsection{Nodemailer}
Nodemailer es un módulo para Node.js que permite enviar emails. Este módulo esta securizado de manera que los mensajes que envía los hace de manera segura. Permite adjuntar elementos a los mensajes y no depende de ningún otro paquete, lo que le convierte en un paquete muy completo.

\subsubsection{Funcionamiento}
Nodemailer es muy sencillo de usar, solo hay que definir el mensaje con los campos y archivos adjuntos que quieras añadir y también habrá que definir el transporter que sera el encargado de enviar el mensaje. En este caso no esta añadido la configuración del transporter.

\begin{codigo}
	var message = \{
		from: 'sender@server.com',
		to: 'receiver@sender.com',
		subject: 'Message title',
		text: 'Plaintext version of the message',
		html: '<p>HTML version of the message</p>'
	\};
	transporter.sendMail(data);
\end{codigo}

\subsubsection{Razón de uso}
Se ha utilizado este paquete para enviar los mails por su sencillez, el gran recorrido que tiene en esta materia, ya que fue creado en 2010 y por su completa documentación.

\subsection{Mongojs}
Mongojs es un paquete para Node.js que permite la conexión con bases de datos MongoDB. Este paquete intenta emular completamente la comunicación directa con la base de datos por lo que su API se asemeja mucho a la de MongoDB.

\subsubsection{Funcionamiento}
Mongojs es muy sencillo de usar, con la siguiente linea de código, conectaremos con una base de datos MongoDB que este en un servidor remoto y si añadimos la variable mycollection, conectaremos directamente con la colección dentro de la base de datos elegida. 

\begin{codigo}
	var db = mongojs('example.com/mydb', ['mycollection'])
\end{codigo}

A continuación se muestra un ejemplo en el que se busca en la base de datos todos los documentos en los que name=Jhon. Esta función devuelve los documentos que cumplan la condición y posteriormente se pueden tratar dentro de la función.

\begin{codigo}
	  db.mycollection.find(\{ 'Name':'Jhon'\}function (err, docs) \{
	  	// Insert code here 
	  \});
\end{codigo}

\subsubsection{Razón de uso}
Se ha utilizado este paquete para conectar con la base de datos por la facilidad con la que conecta con la base de datos y porque emula lo máximo posible la API de mongoDB haciendo su uso muy fácil si ya has utilizado mongoDB con anterioridad.

\subsection{Pdffiller}
Pdffiller es un paquete para Node.js que permite rellenar los formularios de los PDF’s con datos. Su uso se basa en recibir PDF’s con formularios sin rellenar, unos datos y combinarlos de manera que la salida sea un PDF completo. 

\subsubsection{Razón de uso}
Se ha utilizado este paquete por ser el mas utilizado por los usuarios entre las posibilidades. 

\section{MongoDB}
MongoDB es la base de datos NoSQL líder y permite a las empresas ser más ágiles y escalables. Organizaciones de todos los tamaños están usando MongoDB para crear nuevos tipos de aplicaciones, mejorar la experiencia del cliente, acelerar el tiempo de comercialización y reducir costes.\\

Es una base de datos muy ágil por lo que permite cambiar los esquemas al cambiar los requisitos y a la vez proporciona las mismas funcionalidades que se esperan de una base de datos tradicional, manteniendo la velocidad en las búsquedas y siendo consistente.\\

Gracias a ser una base de datos orientada a documentos no hay que ajustarse a un esquema estándar ni obligar a todos los registros a tener la misma información. Esto nos da mucha flexibilidad a la hora de querer introducir nuevos datos o alterar los que ya tenemos.\\

La base de datos del proyecto ha sido poblada mediante JSON, un formato de texto para el intercambio de datos del que hablaremos a continuación. 

\subsection{Razón de uso}
Se ha utilizado esta base de datos para el proyecto por las siguientes razones:

\begin{itemize}
	\item Gracias a su flexibilidad permite alojar diferentes datos en cada documento, permitiendo así ir modificándolos a medida que se van transformando mientras se mantienen unificados.
	\item Gran soporte para proyectos realizados con Node.js y sobre todo con express, lo que hace su integración muy sencilla.
	\item El formato de intercambio de datos que se va a utilizar a lo largo de todo el proyecto será el JSON.
	\item Los limites que oferta la base de datos en cuanto a los documentos encajan con los requisitos del proyecto.
	\item Gratuita y open-source.
\end{itemize}

\section{JSON}
Json (JavaScript Object Notation - Notación de Objetos de JavaScript) es un formato ligero de intercambio de datos. Leerlo y escribirlo es simple para humanos, mientras que para las máquinas es simple interpretarlo y generarlo.

\subsection{Funcionamiento}
Un objeto es un conjunto desordenado de pares nombre/valor. Los objetos comienzan con la llave { (llave apertura) y terminan con la llave } (llave de cierre). Cada nombre es seguido por : (dos puntos) y los pares nombre/valor están separados por , (coma).

\figuraSinMarco{0.8}{imgs/objectJSON.png}{Diagrama de la gramática de un objeto JSON}{objetoJSON}{}

En este ejemplo podemos ver como existe un 2 pares nombre/valor normales y uno que tiene un array de valores dentro.

\figuraSinMarco{0.3}{imgs/JSONejemplo.PNG}{Ejemplo de un JSON}{objetoJSON}{}

\section{Sass (CSS3)}
La definición de Sass y CSS3 integrados es la siguiente:

\begin{quote}
	Sass es un metalenguaje de Hojas de Estilo en Cascada (CSS). Es un lenguaje de script que es traducido a CSS. Sass consiste en dos sintaxis. La sintaxis más reciente, SCSS, usa el formato de bloques como CSS. Éste usa llaves para denotar bloques de código y punto y coma (;) para separar las líneas dentro de un bloque.\\
	
	CSS3 consiste en una serie de selectores y pseudo-selectores que agrupan las reglas que son aplicadas. Sass (en el amplio contexto de ambas sintaxis) extiende CSS proveyendo de varios mecanismos que están presentes en los lenguajes de programación tradicionales, particularmente lenguajes orientados a objetos, pero éste no está disponible para CSS3 como tal. Cuando SassScript se interpreta, éste crea bloques de reglas CSS para varios selectores que están definidos en el fichero SASS. El intérprete de SASS traduce SassScript en CSS. Alternativamente, Sass puede monitorear los ficheros .sass o .scss y convertirlos en un fichero .css de salida cada vez que el fichero .sass o .scss es guardado. Sass es simplemente azúcar sintáctica para escribir CSS.
\end{quote}

Las ventajas que ofrece este metalenguaje son las siguientes:

\begin{itemize}
	\item \textbf{Mixins:} Ofrece la posibilidad de crear bloques de estilo o mixins que se apliquen a mas de una etiqueta o clase, asi, al compilar el código, el estilo se replica a todas las etiquetas y clases aligerando el trabajo.
	\item \textbf{Argumentos:} permite utilizar argumentos para unificar la definición de valores.
	\item \textbf{Herencia:} permite definir una herencia para que los hijos implementen el estilo de los padres. 
\end{itemize}

\subsection{Razón de uso}
Se ha utilizado este metalenguaje en vez de su competidor más claro, less, por las siguientes razones:

\begin{itemize}
	\item Sass genera un código más optimo en algunos de las características que ambos metalenguajes comparten.
	\item Sass tiene varios años más que Less por lo que le hace más robusto y con una mayor comunidad de desarrolladores.
\end{itemize}

Y se usa este metalenguaje en vez de usar CSS directamente por las siguientes razones:

\begin{itemize}
	\item Permite un desarrollo más ágil de las hojas de estilo del proyecto.
	\item Innovación: permite aprender otra herramienta que agilice el desarrollo del CSS.
\end{itemize}

\section{D3.js}
D3.js (o simplemente D3 por las siglas de Data-Driven Documents) es una librería de JavaScript para producir, a partir de datos, infogramas dinámicos e interactivos en navegadores web. En contraste con muchas otras librerías, D3.js permite tener control completo sobre el resultado visual final.\\

Gracias a esta librería se pueden crear gráficos interactivos en los que se unifiquen datos que anteriormente se mostrarían en varios gráficos diferentes, creando así una experiencia más inmersiva en cuanto a la visualización de los datos. \\

Las ventajas principales que ofrece esta librería son:
\begin{itemize}
	\item \textbf{No añade carga a la página:} utiliza lenguajes ya existentes y que no suponen una carga mas para la página web en la que se aloja.
	\item \textbf{Amplias funcionalidades:} permite seleccionar diferentes elementos en una página web y alterarlos. También incluye transiciones para generar cambios visuales, muy similar a JQuery.
	\item \textbf{Asociación de datos:} los datos pueden dirigir la creación de los elementos permitiendo asi a los datos gobernar la visualización y crear diferentes gráficos dependiendo del dataset introducido.
\end{itemize}

\subsection{Funcionamiento}

\begin{codigo}
	codigo scss
\end{codigo}

\figuraSinMarco{0.7}{imgs/scss.png}{Scss y css compilado respectivamente}{scss}{}

\subsection{Razón de uso}
Se ha utilizado esta librería de visualización frente a otras por las siguientes razones:

\begin{itemize}
	\item Innovación: existe un creciente interés de desarrolladores por la herramienta y a su vez falta de programadores con conocimientos sobre ella.
	\item Gran integración con JSON y las herramientas utilizadas en el proyecto.
	\item Open-source y gratuita. 
\end{itemize}


\chapter{Especificación del diseño}

\section{Visión general}
Este capitulo tiene como objetivo describir la labor de diseño realizada para desarrollar el proyecto, así como las herramientas que se han utilizado para realizar estos diseños. En el siguiente listado se describen los diferentes diseños que se van a explicar en este capitulo:

\begin{itemize}
	\item Diseño de la arquitectura: descripción de la arquitectura elegida para el proyecto.
	\item Diseño del servidor: descripción del diseño del servidor.
	\item Diseño de la página web: descripción del diseño de la página web, tanto lógica como visual.
	\item Diseño del widget: descripción del diseño del widget.
	\item Diseño de la base de datos: descripción del diseño de la base de datos y de su estructura.
	\item Diseño del sistema de visualización: descripción del diseño del sistema de visualización y de los gráficos incluidos en este.
\end{itemize}

\section{Herramientas utilizadas}
Las herramientas que se han utilizado a la hora de diseñar el proyecto y sus elementos son las siguientes:

\subsection{Draw.io}
Draw.io es una herramienta web que se utiliza para el desarrollo de diagramas de cualquier tipo, lo cual la hace una herramienta muy potente. Ofrece una gran cantidad de iconos y personalizacion de los mismos para realizar los diagramas lo mas atractivos posibles. Entre sus caracteristicas destaca la integracion con sistemas de almacenamiento online como Google Drive o Dropbox y Github.\\

En lo respectivo a este proyecto, solo se ha utilizado para diseñar los diagramas: AÑADIR DIAGRAMAS.

\section{Diseño de la arquitectura}
La arquitectura del proyecto (ver figura 9.1) se basa en un modelo en tres capas que gira en torno a el servidor Node.js. Estas tres capas permiten separar la parte visual, la lógica y la arquitectura de datos. Gracias a este modelo se puede centralizar la lógica del proyecto en la capa intermedia y abstraerla de los elementos externos a los que ofrece soporte.\\

En la capa de presentación se encuentran la página web y el widget. Estos dos elementos dan soporte a la parte visual del proyecto y están conectados con el servidor para que este les proporcione funcionalidad. En esta capa también se encuentra el sistema de visualización de datos del proyecto, este recibe los datos de la base de datos por medio del servidor que se los provee formateados.\\

En la capa intermedia, la de proceso, se encuentra el servidor que implementa la funcionalidad de toda la solución. Este alberga los widgets que los comercios online van a implementar en sus páginas web y les da la funcionalidad para permitir las donaciones. A la página web le provee del enrutado necesario para implementar sus funcionalidades principales, como la de obtener un certificado de donación o el contacto con el soporte del proyecto. En cuanto al sistema de visualización le otorga los datos, con el formato que necesita, para generar los gráficos deseados. Por ultimo, esta conectado también con la capa de datos, en la que se encuentra la base de datos, con la cual conecta para enviar y pedir datos.\\

Finalmente, en la tercera capa, la de datos, se encuentra el sistema de base de datos. La base de datos, conectada con el servidor, almacena los datos que le llegan desde este y realiza las búsquedas y peticiones que el servidor le pide. Todo esto lo hace mediante el lenguaje de intercambio de datos JSON del que hemos hablado anteriormente.

\figuraSinMarco{0.8}{imgs/Arquitectura.png}{Arquitectura del proyecto}{arquitectura}{}

\section{Diseño del servidor}
El diseño del servidor se ha hecho siguiendo el modelo en el que el servidor Node.js con el apoyo de Express ofrece un API a las diferentes aplicaciones que están conectadas con el. En esta API el servidor ofrece toda la funcionalidad que estas aplicaciones tienen que desarrollar, por demanda de estas, y ademas puede ofrecer mas funcionalidades que el proyecto pueda ofrecer a aplicaciones o elementos que puedan incorporarse en el futuro del proyecto.\\

Gracias a este modelo es muy sencillo añadir nuevas rutas y funcionalidades que las aplicaciones, nuevas o ya existentes, demanden del servidor. Además esta API mantendrá una conexión con la base de datos lo que permitirá el envio, consulta y almacenamiento de información de una manera muy veloz.\\

El servidor estará organizado de manera que en la carpeta \textit{public} estarán alojados todos los archivos comunes incluyendo los scripts de JavaScript que sean necesarios para el desarrollo de las aplicaciones y las librerías correspondientes entre otras cosas\\

Por último cabe destacar que en el diseño creado para el servidor todo gira alrededor del archivo en el que esta desarrollada la API, aunque la creación del servidor en el puerto especificado y la configuración de red se encuentre en el archivo original del servidor.\\

HACER DIAGRAMA DEL SERVIDOR

\section{Diseño de la página web}
El diseño de la página web se ha realizado basándose en las convenciones de Bootstrap para que la web sea totalmente responsiva y visualizable desde la gran mayoría de los navegadores. Gracias al diseño con este framework se podrán añadir los componentes especificos que ofrece e implementar utilidades que sin este framework serian mas complejas de desarrollar.\\

La página web se ha diseñado de manera que toda la información y las funcionalidades disponibles estén a la vista. Por eso se ha planteado una sola vista en la que por medio de algunos botones se desplieguen nuevas ventanas en las que poder añadir alguna funcionalidad o datos que de otra manera alargarían la página de manera negativa. Por otra parte se ha diseñado un menú superior que acompañará al usuario en todo momento para ofrecerle accesos directos a las diferentes secciones de la página web sin tener que ir hasta ella.\\

La página se ha planteado muy minimalista en cuanto a contenido, siguiendo siempre el estilo y colores de Alboan. Se plantea de esta manera para hacer que esta no eclipse a la página principal de Alboan, sino que sea algo complementario que informe mas sobre el proyecto Colmena que los proyectos de la ONG. En cuanto al diseño estético se plantean secciones acotadas en las que se desarrollará una única funcionalidad o se expondrá un tema informativo.\\

En cuanto al diseño de la lógica de la página web, este esta preparado para que la página web no tenga que incluir ningún código JavaScript que no sea estrictamente estético. Toda la lógica de la página web ira integrada en la API que ofrece el servidor.\\

En conclusión, en cuanto a diseño estético la página es altamente escalable ya que solo haría falta añadir una nueva sección. La lógica tampoco supone ningún problema ya que el servicio que la página web quiera dar deberá estar implementado en la API por lo que en la página web solo se deberá desarrollar la llamada a este.\\

AÑADIR DISEÑO PAGINA WEB

\section{Diseño del widget}
El diseño del widget se ha hecho de la manera mas simple posible, utilizando solo tecnologías que todos los navegadores puedan soportar, estas tecnologías son HTML, CSS y JavaScript.\\

Dentro de la lógica del widget se encuentran solo las funciones básicas que el widget tiene que realizar. En este caso se ha pensado en desarrollar solo 3 funciones, las dos primeras se encargarán de hacer el cambio dinámico del dinero en el widget, por lo que no son funcionales, solo visuales. La tercera función si que será la encargada de implementar toda la logica del widget. Esta sera la encargada de enviar la donación al servidor con todos los datos que se necesitan.\\

En cuanto a la estética del widget este se ha diseñado para que sea completamente personalizable, de esta manera, los comercios online puedan diseñar sus propios widgets y los hagan aptos para sus páginas. Para poder mantener esta política de diseño y combinarla con la tecnologías básicas, se ha planteado que el diseño del widget sea guiado mediante un wizard.

\section{Diseño de la base de datos}
La base de datos se ha diseñado partiendo de la premisa de que esta tenia que ser implementada en mongoDB. Una vez analizado los datos que iba a tener que almacenar y viendo los requisitos y necesidades que el proyecto presentaba se ha decidido crear una única base de datos que almacene los documentos en una sola colección.\\

En cuanto a los documentos que la base de datos almacena, estos solo tendrán dos formatos, el primero será el que tienen cuando llegan del widget de donaciones y posteriormente se les añaden los datos de la persona que dona. Por este motivo se diseña la base de datos con una sola colección en la que se alojan los datos de dos formas diferentes.\\

Gracias a este diseño podemos simplificar las conexiones con la base de datos desde el servidor ya que solo tendrá que hacerse a una colección. Por otra parte al solo tener una colección podremos unificar las búsquedas que se hagan en la base de datos. Finalmente al no tener documentos con excesivo tamaño podemos almacenar todos los datos en una sola colección y no corremos el riesgo de que esta se desborde ya que no tiene limite de documentos.

\subsection{Colección}

La colección se llamará 'Donaciones' en la que se albergaran los documentos en los dos formatos que se han descrito anteriormente. Estos documentos tienen los siguientes datos y formatos:

\begin{itemize}
	\item \textbf{Fecha:} la fecha de la donación con dia, mes y año. Esta dividida en un array de tres entero.
	\item \textbf{Usada:} un booleano de si la donacion ha sido usada o no. 
	\item \textbf{IdDonación:} el id de la donación en un entero.
	\item \textbf{Importe:} el importe de la donacion en un entero.
\end{itemize}

Una vez la donación se usa para obtener el certificado de donación, se le añaden al documento en la base de datos los siguientes datos:

\begin{itemize}
	\item \textbf{DNI:} el DNI o CIF de la persona juridica o fisica en un entero.
	\item \textbf{Nombre:} el nombre de la persona juridica o fisica en un String.
	\item \textbf{Razón social:} la razon social de la persona juridica o fisica en un String.
	\item \textbf{Dirección:} la direccion de la persona juridica o fisica en un String.
	\item \textbf{CodigoPostal:} el código postal de la persona juridica o fisica en un entero.
	\item \textbf{Población:} la poblacion de la persona juridica o fisica en un entero.
	\item \textbf{Provincia:} la provincia de la persona juridica o fisica en un entero.
\end{itemize}


\section{Diseño del sistema de visualización}
En el diseño del sistema de visualización se ha tenido en cuenta los datos que se podían conseguir y de que manera representarlos. Teniendo en cuenta el diseño de la base de datos y los datos que se van a almacenar en esta se ha decidido implementar varios gráficos en los que se puede mostrar la mayoría de los datos alojados en la base de datos.\\

Los diseños elegidos para mostrar estos datos serian los siguientes:

\begin{itemize}
	\item \textbf{Sunburst o diagrama circular:} en este gráfico (ver figura 5.2) se muestra un diagrama circular en el que se van clusterizando los datos a medida que vas adentrándote en el. Gracias a este gráfico se pueden analizar sectores mas pequeños de las donaciones y ver la agrupación de diferentes sectores de donantes.
	\figuraSinMarco{0.6}{imgs/sunburst.png}{Diagrama circular}{queso}{}
	\item \textbf{Fechas:} en este gráfico (ver figura 5.3) se muestra un calendario en el que se marcan los días con diferentes colores. Gracias a este gráfico se pueden ver los periodos en los que mas y menos se dona.
	\figuraSinMarco{0.6}{imgs/dategraph.png}{Diagrama de fechas}{fechas}{}
	\item \textbf{Mapa:} en este gráfico se muestra un mapa de España en el que se marcan las zonas en las que ha habido donaciones. Gracias a este gráfico es muy fácil visualizar como es cada zona y que las personas vean en que tipo de comunidad viven.
	\figuraSinMarco{0.6}{imgs/sunburst.png}{Diagrama circular}{arquitectura}{}
\end{itemize} 

El sistema de visualización de datos se ha planteado para que ejerza la minima carga al sistema, por lo que se plantea con una librería de visualización para la creación de infogramas interactivos y archivos JSON para poblarla, de manera que la tarea de formatear los datos para integrarlos en la librería será mínima.



\chapter{Consideraciones sobre la implementación}

\section{Visión general}
Este capítulo contiene los aspectos más significativos sobre la implementación del sistema y sus funcionalidades.El contenido central de este capitulo se centra en explicar que pasos se han dado a la hora de desarrollar el sistema y como se ha organizado este, simplificando la labor de comprensión del sistema para terceros. Por último, al principio del capitulo se explicarán las herramientas que se han utilizado a la hora de desarrollar el sistema. 
\section{Entorno de desarrollo}
Las herramientas que se han utilizado para facilitar el desarrollo y la arquitectura del proyecto han sido las siguientes.

\subsection{Atom}
Atom se define a si mismo como ``el editor de textos hackable del siglo XXI''. Dentro de este editor de texto o entorno de desarrollo integrado (IDE), ya que cumple todos los requisitos de uno de estos, se encuentran una serie de funcionalidades y características que lo hacen una de las herramientas mas potentes del mercado. Dentro de las funcionalidades que ofrece esta herramienta se encuentran los paquetes que la comunidad crea y desarrolla, gracias a estos paquetes el programa a conseguido implementar la corrección y sintaxis de muchos lenguajes. Por otra parte, este editor de textos es totalmente personalizable ya que esta desarrollado con tecnologias web y ofrece una IU totalmente modificable mediante CSS/Less. Por ultimo, y no menos importante, Atom esta desarrollado por GitHub, por lo que es open-source y ofrece la posibilidad de ayudar en el desarrollo del código de la aplicación.

\subsection{Brackets}
Brackets es un editor de texto moderno enfocado en el diseño de la parte visual de las aplicaciones web. La funcionalidad mas destacada de este software es la vista previa que permite al desarrollador ver en directo como quedarán los cambios que realice en el código, directamente en la página web, esto ahorra tiempo al no tener que recargar la web constantemente.Brackets también ofrece un editor interno que permite navegar por todos los archivos CSS que una etiqueta implemente, lo que aligera la carga de estar cambiando entre las diferentes hojas de estilo. Por ultimo, entre sus funcionalidades tambien destaca el soporte a los preprocesadores, brackets permite hacer cambios directamente en los archivos de las hojas de estilo de los preprocesadores y ver esos cambios directamente, sin tener que compilarlos.

\subsection{Git}
Git es un sistema de control de versiones distribuido gratuito y de código abierto que está diseñado para manejar desde pequeños proyectos particulares a proyectos de grandes organizaciones con una gran velocidad y eficiencia. Además es muy fácil de aprender, tiene una excelente documentación y al ser usado por muchísimos desarrolladores tiene una gran comunidad de usuarios dispuestos a resolver cualquier problema.

\section{Implementación del servidor}

\section{Implementación de la página web}
La página web esta implementada sobre una plantilla de Bootstrap lo que ha permitido que el desarrollo se haga mas agil. En cambio, la plantilla ha sido alterada para que el diseño encaje con el de la ONG, por lo que de la plantilla original solo queda la estructura. \\

La página web esta dividida por plantillas que permiten 
\section{Implementación del widget}

\section{Implementación de la base de datos}

\chapter{Plan de pruebas}

Durante el desarrollo del proyecto se han ido realizando diferentes pruebas que han permitido que el proyecto avance de manera regular, en caso de haber algún error se ha planificado su corrección para continuar con el correcto desarrollo del proyecto. Durante este capítulo se explicarán algunas de las pruebas realizadas.

\section{Pruebas del servidor}
Las pruebas del servidor se han realizado mediante una clase que Node.js lleva implementada internamente, la clase \textit{Console}. Esta clase nos permite loguear todos los eventos o mensajes de la aplicación. Su funcionamiento es similar al famoso framework Log4J. La clase \textit{Console} permite mostrar el mensaje que aparece por consola tratándolo de manera diferente dependiendo de si es un mensaje de error, de log, de alerta...\\

Por otra parte, las pruebas unitarias se han realizado con un framework llamado Mocha\cite{mocha} que permite simular peticiones HTTP e incluso testear funciones asíncronas mediante asertos, como es el caso de las consultas a la base de datos.

\section{Pruebas a la base de datos}
Las pruebas a la base de datos se han hecho en materia de velocidad de peticiones y almacenamiento de los datos ya que al ser una base de datos noSQL no podemos cometer el error de añadir datos que no correspondan a las tablas configuradas.\\

Para estas pruebas se ha creado un Script que genera donaciones de manera automatizada y posteriormente se han hecho peticiones a la base de datos para medir su respuesta. La base de datos tarda en añadir 100.000 registros, 49,95 segundos. Y tarda en recuperar uno de los registros con un numero de donación aleatorio 0,71 segundos.\\
\newpage
\codigofuente{Python}{Script para introducir datos a la base de datos}{src/dataset.py}


\chapter{Manual de usuario}

En este capitulo se muestra los manuales para utilizar el sistema paso a paso. Los manuales explicados en el capitulo son los siguientes:

\begin{itemize}
	\item \textbf{Manual del donante:} en este manual se incluyen todos los pasos que una persona debe hacer desde que realiza la donación hasta que recibe el certificado de donación.
	\item \textbf{Manual del comercio online:} en este manual se incluyen todos los pasos que una empresa debe seguir para añadir el widget solidario a su comercio online.
\end{itemize}

\section{Manual del donante}
Partimos desde el punto en el que el usuario se encuentra en un comercio online y se dispone a comprar lo que el desee, en este manual se ha utilizado la tienda de Alboan y el usuario esta comprando tarjetas de navidad. 


Una vez añadido el euro solidario a su carrito puede finalizar su compra. Una vez finalizada la compra el widget se encargará de enviar un mail a la cuenta de correo que tienes en la página web donde realizaste la compra. \\

FOTO TIENDA CON WIDGET

Una vez tengamos ese numero en el mail nos acercamos a la página web de la Colmena. Y seleccionamos la opción \textit{Quiero recibir un certificado} o bajamos por la página web hasta llegar a esa sección.\\

\figuraSinMarco{1}{imgs/principal.png}{Primera vista al entrar a la web}{principal}{}

Clicamos en el boton donde pone \textit{Obtener certificado} y se nos desplegará una ventana nueva en la que podremos introducir nuestro numero de donacion. En el caso de que nuestro numero haya sido usado o de que lo introduzcamos incorrectamente nos aparecerá un mensaje de aviso que nos indicará el error. \\

\figuraSinMarco{0.8}{imgs/numcertificado.PNG}{Campo para introducir el numero de donación}{numcertificado}{}

Una vez introducido nuestro numero de donacion correctamente la web nos dirigira a una nueva pagina en la que se nos mostrara un formulario. Dentro del formulario debemos introducir los datos fiscales para crear el certificado de donación. El formulario comprueba que los datos que hayamos introducido sean validos, en este caso el unico dato que tiene una restriccion es el email. \\

\figuraSinMarco{1}{imgs/formularioCertificado.PNG}{Formulario para obtener certificado con fallo en el email}{formulario}{}

Finalmente si hemos introducido nuestros datos correctamente nos llegará al correo indicado un mensaje en el que ira adjunto nuestro certificado de donación. El certificado de donacion se puede utilizar para fines fiscales. El certificado esta añadido en el capitulo de anexos.\\

Dentro de la página web también se puede obtener información sobre los diferentes proyectos que la ONG tiene abiertos en estos momentos. Para ver estos proyectos podemos navegar hasta la seccion donde pone \textit{Proyectos} o clicar en la opcion de \textit{Quiero conocer proyectos} que ofrece el menu de la parte superior. Una vez en esta seccion podemos clicar encima de cualquiera de los proyectos que se nos ofrecen, entonces se desplegará una nueva ventana en la que se nos ofrecerá mas informacion sobre el proyecto.

\figuraSinMarco{1}{imgs/mujeresProyecto.PNG}{Seccion de una ventana con informacion de un proyecto apoyado por la Colmena}{mujeres}{}

\section{Manual del comercio online}
En el caso de ser una empresa que quiere implementar el widget en su comercio online se tendrian que seguir estos pasos. En primer lugar habria que acceder a la página web de la colmena y acceder a la seccion donde pone \textit{Creación} o clicar en la opcion donde aparece \textit{Quiero crear un widget} del menu superior.\\

Una vez aqui clicamos en el boton donde aparece \textit{Crear widget} y nos aparecerá una ventana con un asistente en la que podremos crear el widget a nuestra eleccion. Esto es solo la demo para saber como se pueden diseñar los widgets. En el caso de querer crear un widget real deberás ponerte en contacto con el equipo de la Colmena.\\

\figuraSinMarco{1}{imgs/wizard1.PNG}{Demo del asistente para la creacion de nuevos widgets}{demowizard}{}

Bajamos hasta la ultima seccion de la página web o clicamos en la opcion donde pone \textit{Quiero contactar con vosotros}. Una vez en esta seccion podemos rellenar el formulario indicando que queremos crear un nuevo widget para nuestra página web, es entonces cuando el equipo de la Colmena se pondra en contacto con nosotros y nos enviara una ruta que nos llevará al asistente de creacion de widgets.\\

\figuraSinMarco{1}{imgs/contacto.PNG}{Formulario de contacto con el equipo de la Colmena}{contacto}{}

\figuraSinMarco{0.8}{imgs/mensaje.PNG}{Mail recibido por el equipo de la Colmena}{mail}{}

Por ultimo accedemos a la página web que se nos suministra por correo electronico y volvemos a estar delante del asistente que nos permite crear el el widget solidario para nuestra página web. Al saber como funciona podemos diseñar agilmente el widget. 

\figuraSinMarco{1}{imgs/wizard2.PNG}{Asistente de creación para nuevos widgets}{wizard2}{}

Finalmente ya tenemos nuestro widget creado y lo podremos implementar en nuestra página web mediante el manual que la Colmena nos ofrece. Este manual tiene dos partes, la de la implementación en el comercio online y la que tiene que realizar la Colmena en su servidor online. El manual esta ubicado en los anexos.

\chapter{Incidencias}

\section{Interacción con el cliente}

\chapter{Conclusiones y lineas futuras}

Este capitulo contiene las conclusiones que se han obtenido después de haber desarrollado el proyecto y las posibles acciones que se podrían tomar en el futuro para mejorarlo o integrarlo con el sistema que existe en Alboan.

\section{Conclusiones}
Al comienzo del proyecto este se presentaba sin una definición clara ya que Alboan sabia que quería un nuevo sistema para las micro donaciones y que quería que estuviese implementado con un widget que soportase las micro donaciones. Junto a ellos se consiguió definir que la solución incluiría una página web que publicitase el proyecto. Finalmente y cuando el proyecto tuvo mas forma se decidió añadirle el sistema de visualización de datos para crear los gráficos y estadísticas que Alboan posteriormente podría utilizar. Es por esto que el proyecto ha tenido unos requisitos un poco cambiantes y se le ha permitido al cliente cambiar las historias de cliente.\\

El desarrollo ha sido fluido y el proyecto se ha ido completando con normalidad. En varias ocasiones ha habido dudas sobre como desarrollar algunas de las funcionalidades pero con ayuda del director de proyecto, que me puso en contacto con personas mas experimentadas en el tema, pude resolverlas sin ningún problema.\\

Gracias al proyecto he tenido la oportunidad de aprender a gestionar proyectos por lo que me he interesado por las diferentes metodologías que existen. En cuanto a la decision de las tecnologias he decido interesarme mas por las nuevas metodologías ágiles que me han permitido aprender nuevos modelos de gestion de proyectos. Esta metodologia, scrum, ya las habia aplicado en una asignatura de clase, pero en un entorno muy controlado, al llevarla a la practica me he dado cuenta que la teoria no se aplica del todo en la practica.\\

Por otra parte el proyecto me ha exigido formación en muchas tecnologias que no habia utilizado anteriormente. Gracias a este he tenido que que investigar en las tecnologias web actuales y me ha exigido formarme tanto en Node.js como en los paquetes que ofrece mediante \textit{npm}. Por otra parte, de las tecnologías básicas del desarrollo web también he tenido que formarme ya que al intentar desarrollar en HTML5 o utilizar un precompilador para las hojas de estilo he adquirido mucha formación en estos temas.\\

En conclusión, el proyecto ha sido una oportunidad para prepararme para lo que puede ser mi futuro laboral, tanto en la gestión de proyectos como en el desarrollo de los diferentes elementos del sistema completo. Al tener que interpretar varios roles en el desarrollo del proyecto he visto cuales son las exigencias y alcance de cada uno, lo que me ha permitido discernir sobre varias opciones de mi futuro laboral.

\section{Lineas futuras}
El avance mas importante que se podría hacer de aquí en adelante seria conseguir integrar el proyecto con la infraestructura de Alboan. Alboan tiene subcontratada a una empresa que le lleva toda la infraestructura tecnológica y la comunicación con ellos no ha sido buena ya que no estaban seguros de si integrar el proyecto por miedo a que Alboan no lo aceptase del todo. \\

A parte de esta mejora principal se podrán implementar nuevas lineas que permitirán que el proyecto sea de mas calidad:

\begin{itemize}
	\item Implementar i18n en la página web para que sea mas accesible mediante \textit{i18n-express}
	\item Implementar un sistema de comentarios que los demás usuarios puedan ver para recoger las opiniones.
\end{itemize}

% Apéndices
\backmatter
\appendix


%\chapter{Licencia}
%\begin{center}
                    GNU GENERAL PUBLIC LICENSE\\
                       Version 2, June 1991\\*[2ex]

 Copyright (C) 1989, 1991 Free Software Foundation, Inc.\\
     59 Temple Place, Suite 330, Boston, MA  02111-1307  USA\\
 Everyone is permitted to copy and distribute verbatim copies\\
 of this license document, but changing it is not allowed.
\end{center}

\begin{center}
Preamble
\end{center}

  The licenses for most software are designed to take away your
freedom to share and change it.  By contrast, the GNU General Public
License is intended to guarantee your freedom to share and change free
software--to make sure the software is free for all its users.  This
General Public License applies to most of the Free Software
Foundation's software and to any other program whose authors commit to
using it.  (Some other Free Software Foundation software is covered by
the GNU Library General Public License instead.)  You can apply it to
your programs, too.

  When we speak of free software, we are referring to freedom, not
price.  Our General Public Licenses are designed to make sure that you
have the freedom to distribute copies of free software (and charge for
this service if you wish), that you receive source code or can get it
if you want it, that you can change the software or use pieces of it
in new free programs; and that you know you can do these things.

  To protect your rights, we need to make restrictions that forbid
anyone to deny you these rights or to ask you to surrender the rights.
These restrictions translate to certain responsibilities for you if you
distribute copies of the software, or if you modify it.

  For example, if you distribute copies of such a program, whether
gratis or for a fee, you must give the recipients all the rights that
you have.  You must make sure that they, too, receive or can get the
source code.  And you must show them these terms so they know their
rights.

  We protect your rights with two steps: (1) copyright the software, and
(2) offer you this license which gives you legal permission to copy,
distribute and/or modify the software.

  Also, for each author's protection and ours, we want to make certain
that everyone understands that there is no warranty for this free
software.  If the software is modified by someone else and passed on, we
want its recipients to know that what they have is not the original, so
that any problems introduced by others will not reflect on the original
authors' reputations.

  Finally, any free program is threatened constantly by software
patents.  We wish to avoid the danger that redistributors of a free
program will individually obtain patent licenses, in effect making the
program proprietary.  To prevent this, we have made it clear that any
patent must be licensed for everyone's free use or not licensed at all.

  The precise terms and conditions for copying, distribution and
modification follow.

\begin{center}
		    GNU GENERAL PUBLIC LICENSE\\
   TERMS AND CONDITIONS FOR COPYING, DISTRIBUTION AND MODIFICATION
\end{center}

  0. This License applies to any program or other work which contains
a notice placed by the copyright holder saying it may be distributed
under the terms of this General Public License.  The "Program", below,
refers to any such program or work, and a "work based on the Program"
means either the Program or any derivative work under copyright law:
that is to say, a work containing the Program or a portion of it,
either verbatim or with modifications and/or translated into another
language.  (Hereinafter, translation is included without limitation in
the term "modification".)  Each licensee is addressed as "you".

\noindent
Activities other than copying, distribution and modification are not
covered by this License; they are outside its scope.  The act of
running the Program is not restricted, and the output from the Program
is covered only if its contents constitute a work based on the
Program (independent of having been made by running the Program).
Whether that is true depends on what the Program does.

  1. You may copy and distribute verbatim copies of the Program's
source code as you receive it, in any medium, provided that you
conspicuously and appropriately publish on each copy an appropriate
copyright notice and disclaimer of warranty; keep intact all the
notices that refer to this License and to the absence of any warranty;
and give any other recipients of the Program a copy of this License
along with the Program.

\noindent
You may charge a fee for the physical act of transferring a copy, and
you may at your option offer warranty protection in exchange for a fee.

  2. You may modify your copy or copies of the Program or any portion
of it, thus forming a work based on the Program, and copy and
distribute such modifications or work under the terms of Section 1
above, provided that you also meet all of these conditions:

    a) You must cause the modified files to carry prominent notices
    stating that you changed the files and the date of any change.

    b) You must cause any work that you distribute or publish, that in
    whole or in part contains or is derived from the Program or any
    part thereof, to be licensed as a whole at no charge to all third
    parties under the terms of this License.

    c) If the modified program normally reads commands interactively
    when run, you must cause it, when started running for such
    interactive use in the most ordinary way, to print or display an
    announcement including an appropriate copyright notice and a
    notice that there is no warranty (or else, saying that you provide
    a warranty) and that users may redistribute the program under
    these conditions, and telling the user how to view a copy of this
    License.  (Exception: if the Program itself is interactive but
    does not normally print such an announcement, your work based on
    the Program is not required to print an announcement.)

\noindent
These requirements apply to the modified work as a whole.  If
identifiable sections of that work are not derived from the Program,
and can be reasonably considered independent and separate works in
themselves, then this License, and its terms, do not apply to those
sections when you distribute them as separate works.  But when you
distribute the same sections as part of a whole which is a work based
on the Program, the distribution of the whole must be on the terms of
this License, whose permissions for other licensees extend to the
entire whole, and thus to each and every part regardless of who wrote it.

\noindent
Thus, it is not the intent of this section to claim rights or contest
your rights to work written entirely by you; rather, the intent is to
exercise the right to control the distribution of derivative or
collective works based on the Program.

\noindent
In addition, mere aggregation of another work not based on the Program
with the Program (or with a work based on the Program) on a volume of
a storage or distribution medium does not bring the other work under
the scope of this License.

  3. You may copy and distribute the Program (or a work based on it,
under Section 2) in object code or executable form under the terms of
Sections 1 and 2 above provided that you also do one of the following:
\begin{quote}
    a) Accompany it with the complete corresponding machine-readable
    source code, which must be distributed under the terms of Sections
    1 and 2 above on a medium customarily used for software interchange; or,

    b) Accompany it with a written offer, valid for at least three
    years, to give any third party, for a charge no more than your
    cost of physically performing source distribution, a complete
    machine-readable copy of the corresponding source code, to be
    distributed under the terms of Sections 1 and 2 above on a medium
    customarily used for software interchange; or,

    c) Accompany it with the information you received as to the offer
    to distribute corresponding source code.  (This alternative is
    allowed only for noncommercial distribution and only if you
    received the program in object code or executable form with such
    an offer, in accord with Subsection b above.)
\end{quote}
\noindent
The source code for a work means the preferred form of the work for
making modifications to it.  For an executable work, complete source
code means all the source code for all modules it contains, plus any
associated interface definition files, plus the scripts used to
control compilation and installation of the executable.  However, as a
special exception, the source code distributed need not include
anything that is normally distributed (in either source or binary
form) with the major components (compiler, kernel, and so on) of the
operating system on which the executable runs, unless that component
itself accompanies the executable.

\noindent
If distribution of executable or object code is made by offering
access to copy from a designated place, then offering equivalent
access to copy the source code from the same place counts as
distribution of the source code, even though third parties are not
compelled to copy the source along with the object code.

  4. You may not copy, modify, sublicense, or distribute the Program
except as expressly provided under this License.  Any attempt
otherwise to copy, modify, sublicense or distribute the Program is
void, and will automatically terminate your rights under this License.
However, parties who have received copies, or rights, from you under
this License will not have their licenses terminated so long as such
parties remain in full compliance.

  5. You are not required to accept this License, since you have not
signed it.  However, nothing else grants you permission to modify or
distribute the Program or its derivative works.  These actions are
prohibited by law if you do not accept this License.  Therefore, by
modifying or distributing the Program (or any work based on the
Program), you indicate your acceptance of this License to do so, and
all its terms and conditions for copying, distributing or modifying
the Program or works based on it.

  6. Each time you redistribute the Program (or any work based on the
Program), the recipient automatically receives a license from the
original licensor to copy, distribute or modify the Program subject to
these terms and conditions.  You may not impose any further
restrictions on the recipients' exercise of the rights granted herein.
You are not responsible for enforcing compliance by third parties to
this License.

  7. If, as a consequence of a court judgment or allegation of patent
infringement or for any other reason (not limited to patent issues),
conditions are imposed on you (whether by court order, agreement or
otherwise) that contradict the conditions of this License, they do not
excuse you from the conditions of this License.  If you cannot
distribute so as to satisfy simultaneously your obligations under this
License and any other pertinent obligations, then as a consequence you
may not distribute the Program at all.  For example, if a patent
license would not permit royalty-free redistribution of the Program by
all those who receive copies directly or indirectly through you, then
the only way you could satisfy both it and this License would be to
refrain entirely from distribution of the Program.

\noindent
If any portion of this section is held invalid or unenforceable under
any particular circumstance, the balance of the section is intended to
apply and the section as a whole is intended to apply in other
circumstances.

\noindent
It is not the purpose of this section to induce you to infringe any
patents or other property right claims or to contest validity of any
such claims; this section has the sole purpose of protecting the
integrity of the free software distribution system, which is
implemented by public license practices.  Many people have made
generous contributions to the wide range of software distributed
through that system in reliance on consistent application of that
system; it is up to the author/donor to decide if he or she is willing
to distribute software through any other system and a licensee cannot
impose that choice.

\noindent
This section is intended to make thoroughly clear what is believed to
be a consequence of the rest of this License.

  8. If the distribution and/or use of the Program is restricted in
certain countries either by patents or by copyrighted interfaces, the
original copyright holder who places the Program under this License
may add an explicit geographical distribution limitation excluding
those countries, so that distribution is permitted only in or among
countries not thus excluded.  In such case, this License incorporates
the limitation as if written in the body of this License.

  9. The Free Software Foundation may publish revised and/or new versions
of the General Public License from time to time.  Such new versions will
be similar in spirit to the present version, but may differ in detail to
address new problems or concerns.

Each version is given a distinguishing version number.  If the Program
specifies a version number of this License which applies to it and "any
later version", you have the option of following the terms and conditions
either of that version or of any later version published by the Free
Software Foundation.  If the Program does not specify a version number of
this License, you may choose any version ever published by the Free Software
Foundation.

  10. If you wish to incorporate parts of the Program into other free
programs whose distribution conditions are different, write to the author
to ask for permission.  For software which is copyrighted by the Free
Software Foundation, write to the Free Software Foundation; we sometimes
make exceptions for this.  Our decision will be guided by the two goals
of preserving the free status of all derivatives of our free software and
of promoting the sharing and reuse of software generally.

\begin{center}
			    NO WARRANTY
\end{center}

  11. BECAUSE THE PROGRAM IS LICENSED FREE OF CHARGE, THERE IS NO WARRANTY
FOR THE PROGRAM, TO THE EXTENT PERMITTED BY APPLICABLE LAW.  EXCEPT WHEN
OTHERWISE STATED IN WRITING THE COPYRIGHT HOLDERS AND/OR OTHER PARTIES
PROVIDE THE PROGRAM "AS IS" WITHOUT WARRANTY OF ANY KIND, EITHER EXPRESSED
OR IMPLIED, INCLUDING, BUT NOT LIMITED TO, THE IMPLIED WARRANTIES OF
MERCHANTABILITY AND FITNESS FOR A PARTICULAR PURPOSE.  THE ENTIRE RISK AS
TO THE QUALITY AND PERFORMANCE OF THE PROGRAM IS WITH YOU.  SHOULD THE
PROGRAM PROVE DEFECTIVE, YOU ASSUME THE COST OF ALL NECESSARY SERVICING,
REPAIR OR CORRECTION.

  12. IN NO EVENT UNLESS REQUIRED BY APPLICABLE LAW OR AGREED TO IN WRITING
WILL ANY COPYRIGHT HOLDER, OR ANY OTHER PARTY WHO MAY MODIFY AND/OR
REDISTRIBUTE THE PROGRAM AS PERMITTED ABOVE, BE LIABLE TO YOU FOR DAMAGES,
INCLUDING ANY GENERAL, SPECIAL, INCIDENTAL OR CONSEQUENTIAL DAMAGES ARISING
OUT OF THE USE OR INABILITY TO USE THE PROGRAM (INCLUDING BUT NOT LIMITED
TO LOSS OF DATA OR DATA BEING RENDERED INACCURATE OR LOSSES SUSTAINED BY
YOU OR THIRD PARTIES OR A FAILURE OF THE PROGRAM TO OPERATE WITH ANY OTHER
PROGRAMS), EVEN IF SUCH HOLDER OR OTHER PARTY HAS BEEN ADVISED OF THE
POSSIBILITY OF SUCH DAMAGES.

\begin{center}
		     END OF TERMS AND CONDITIONS

	    How to Apply These Terms to Your New Programs
\end{center}

  If you develop a new program, and you want it to be of the greatest
possible use to the public, the best way to achieve this is to make it
free software which everyone can redistribute and change under these terms.

  To do so, attach the following notices to the program.  It is safest
to attach them to the start of each source file to most effectively
convey the exclusion of warranty; and each file should have at least
the "copyright" line and a pointer to where the full notice is found.

\begin{quote}
    $<$one line to give the program's name and a brief idea of what it does.$>$\\
    Copyright (C) $<$year$>$  $<$name of author$>$

    This program is free software; you can redistribute it and/or modify
    it under the terms of the GNU General Public License as published by
    the Free Software Foundation; either version 2 of the License, or
    (at your option) any later version.

    This program is distributed in the hope that it will be useful,
    but WITHOUT ANY WARRANTY; without even the implied warranty of
    MERCHANTABILITY or FITNESS FOR A PARTICULAR PURPOSE.  See the
    GNU General Public License for more details.

    You should have received a copy of the GNU General Public License
    along with this program; if not, write to the Free Software
    Foundation, Inc., 59 Temple Place, Suite 330, Boston, MA  02111-1307  USA
\end{quote}

\noindent
Also add information on how to contact you by electronic and paper mail.

\noindent
If the program is interactive, make it output a short notice like this
when it starts in an interactive mode:

\begin{quote}
    Gnomovision version 69, Copyright (C) year  name of author
    Gnomovision comes with ABSOLUTELY NO WARRANTY; for details type `show w'.
    This is free software, and you are welcome to redistribute it
    under certain conditions; type `show c' for details.
\end{quote}

\noindent
The hypothetical commands `show w' and `show c' should show the appropriate
parts of the General Public License.  Of course, the commands you use may
be called something other than `show w' and `show c'; they could even be
mouse-clicks or menu items--whatever suits your program.

\noindent
You should also get your employer (if you work as a programmer) or your
school, if any, to sign a ``copyright disclaimer'' for the program, if
necessary.  Here is a sample; alter the names:

\begin{quote}
  Yoyodyne, Inc., hereby disclaims all copyright interest in the program\\
  `Gnomovision' (which makes passes at compilers) written by James Hacker.\\*[2ex]

  $<$signature of Ty Coon$>$, 1 April 1989\\
  Ty Coon, President of Vice
\end{quote}

\noindent
This General Public License does not permit incorporating your program into
proprietary programs.  If your program is a subroutine library, you may
consider it more useful to permit linking proprietary applications with the
library.  If this is what you want to do, use the GNU Library General
Public License instead of this License.


\bibliografia{otrasreferencias}
%\bibliografiaOtras{otrasreferencias.bib} %Opcional
%\chapter{Acrónimos}


\end{document}
